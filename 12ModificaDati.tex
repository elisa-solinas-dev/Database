\chapter{Modifica dei dati}
La parte di DML (Data Manipulation Language) comprende i comandi per interrogare e modificare il contenuto della base di dati.



\section{Inserimento}
Il comando di inserimento di righe nella base di dati presenta due sintassi alternative: la prima permette di inserire \textbf{singole righe} all'interno delle tabelle. L'argomento della clausola \texttt{values} rappresenta esplicitamente i valori degli attributi della singola riga.
	\begin{lstlisting}
		insert into NomeTabella [ListaAttributi]
			values (ListaValori)
	\end{lstlisting}
La seconda forma, invece, permette di aggiungere \textbf{insiemi di righe}, estratti dal contenuto della base di dati.
	\begin{lstlisting}
		insert into NomeTabella [ListaAttributi]
			select ...
	\end{lstlisting}
Ciascuna forma del comando possiede uno specifico campo di applicazione: la prima forma è quella tipicamente usata all'interno dei programmi per riempire una tabella con i dati forniti direttamente dagli utenti.\\
Ogni uso del comando di \texttt{insert} è generalmente associato al riempimento di una \textbf{maschera} (o \textbf{form}), ovvero un'interfaccia di facile uso in cui all'utente vengono presentati sul video il nome dei vari attributi e appositi spazi in cui immettere i relativi valori.\\
La seconda forma permette invece di inserire dati in una tabella a partire da altre informazioni presenti nella base di dati.\\\\
Se in un inserimento non vengono specificati i valori di tutti gli attributi della tabella, agli attributi mancanti viene assegnato il valore di default, o in assenza di questo, il valore nullo;  se l'inserimento viola un vincolo \texttt{not null} definito sull'attributo, l'inserimento viene rifiutato.\\\\
Si noti, infine, che la corrispondenza tra gli attributi della tabella e i valori da inserire è data dall'ordine in cui compaiono i termini nella definizione della tabella. Perciò, al primo attributo che compare in \texttt{ListaValori} (per la prima forma del comando) o al primo elemento della clausola \texttt{select} (per la seconda forma) deve corrispondere il primo attributo che compare in \texttt{ListaAttributi} (o nella definizione della tabella se \texttt{ListaAttributi} è omesso), e così via per gli altri attributi.

	

\section{Cancellazione}
Il comando \texttt{delete} elimina righe dalle tabelle della base di dati, seguendo la semplice sintassi:
	$$\texttt{delete from NomeTabella where [Condizione]}$$
Quando la condizione argomento della clausola \texttt{where} non viene specificata, il comando cancella tutte le righe della tabella,, altrimenti vengono rimosse solo le righe che soddisfano la condizione.\\
Qualora esista un vincolo di integrità referenziale con politica di \texttt{cascade} in cui la tabella viene referenziata, allora la cancellazione di righe dalla tabella può comportare la cancellazione di righe appartenenti ad altre tabelle (e si può generare una reazione a catena se queste cancellazioni a loro volta causano la cancellazione di righe di altre tabelle).



\section{Modifica}
Il comando \texttt{update} presenta una sintassi più complicata rispetto agli altri due:
	\begin{lstlisting}
		update NomeTabella
			set Attributo = <Espressione | select ...SQL... | null | default>
			{, Attributo = <Espressione | select ...SQL... | null | default> }
		[where Condizione]
	\end{lstlisting}
Il comando \texttt{update} permette di aggiornare uno o più attributi delle righe di \texttt{NomeTabella} che soddisfano l'eventuale \texttt{Condizione}.\\
Se il comando non presenta la clausola \texttt{where}, si suppone che la condizione sia soddisfatta e si esegue la modifica su tutte le righe. Il nuovo valore cui viene posto l'attributo può essere:
	\begin{enumerate}
			\item {Il risultato della valutazione di un'espressione sugli attributi della tabella, che può anche far riferimento al valore corrente dell'attributo che verrà modificato dal comando.}
			\item {Il risultato di una generica interrogazione SQL}
			\item {Il valore nullo}
			\item {Il valore di default per il dominio}
	\end{enumerate}
La natura \textbf{set-oriented} di SQL presenta alcune particolarità di cui è necessario tenere conto.\\
Consideriamo  lo schema 
		\begin{equation}\begin{aligned}
			IMPIEGATO (\underline{Nome, Cognome}, Dipartimento, Ufficio, Stipendio, Citta)\\
			DIPARTIMENTO (\underline{Nome}, Indirizzo, Citta)
		\end{aligned}\end{equation}
Vogliamo modificare gli stipendi dei dipendenti, aumentando del $10\%$ gli stipendi inferiori a 30, e del $15\%$ agli stipendi superiori.\\
Un modo per aggiornare in questo modo la base dati consiste nell'eseguire questo comando:
	\begin{lstlisting}
		update Impiegato
			set I.Stipendio = I.Stipendio * 1.1
			where I.Stipendio <= 30
			
		update Impiegato
			set I.Stipendio = I.Stipendio * 1.15
			where I.Stipendio > 30
	\end{lstlisting}
Il problema di questa soluzione è che se consideriamo un dipendente con uno stipendio iniziale di 30, questo soddisferà la condizione del primo comando di aggiornamento, per cui l'attributo \texttt{Stipendio} verrà posto pari a 33. Ma, a questo punto, la riga soddisferà anche le condizioni del secondo comando di aggiornamento, per cui lo stipendio sarà nuovamente modificato. \\
Il risultato finale è che per questa riga l'aumento complessivo sarà del $26.5 \%$, violando quindi i requisiti di partenza.\\\\
Il problema ha origine nel carattere \textit{set-oriented} di SQL. Con un linguaggio \textit{tuple-oriented} sarebbe possibile analizzare le righe una a una e applicare o l'una o l'altra delle modifiche a seconda del valore dello stipendio.\\
In questo caso una semplice soluzione consiste nell'invertire l'ordine di esecuzione dei due comandi, aumentando prima gli stipendi superiori e poi i rimanenti.\\
In casi più complicati risulta necessario introdurre degli aggiornamenti intermedi, fare uso di costrutti avanzati, oppure cambiare completamente approccio e scrivere un programma in un tradizionale linguaggio di programmazione. 