\chapter{Algebra relazionale}

Il modello relazionale descrive il modello del \textbf{DDL} (Data Definition Language), mentre l'algebra relazionale si occupa del \textbf{DML} (Data Manipulation Language), che manipola le tuple presenti nella base dati.\\\\
Il cuore del DML è il linguaggio di interrogazione.\\
Secondo il modello dell'interrogazione di Codd, ciascuna interrogazione ha come input la base dati e come output una relazione.\\\\
\textbf{ES:} elencare i pazienti ricoverati nel reparto il cui primario è il Dott. Neri; visualizzare cognome, nome e nome del reparto.\\\\
Codd formalizza l'interrogazione utilizzando il paradigma algebrico, cioè una costruzione \textbf{procedurale} dell'interrogazione: ciascuna interrogazione elenca i passi per eseguire l'operazione.\\
La presentazione è però più astratta, in quanto utilizza degli operatori algebrici.\\
Gli operatori dell'algebra relazionale ricevono, come argomenti, relazioni e producono in uscita altre relazioni (dette anche \textbf{relazioni virtuali}).

\section{Operatore selezione}
Data una relazione $r$ su uno schema $A$, l'operatore di selezione viene indicato con
    \begin{equation}\begin{aligned}
        \sigma_{p}(r(A))
    \end{aligned}\end{equation}
Dove $p$ è un predicato e $r(A)$ è l'argomento dell'operatore.\\
La selezione produce una relazione senza nome che ha l'identico schema $A$ della relazione argomento, ma solo le tuple che soddisfano il predicato $p$.

\subsection{Sintassi del predicato di selezione}
Il predicato $p$ è un'espressione booleana di predicati atomici, che possono essere solo di due tipi: confronto tra due attributi, oppure confronto tra un attributo e una costante.

\subsection{Cardinalità della selezione}
La cardinalità della relazione virtuale prodotta dall'operazione di selezione $\sigma_p(r(A))$ è 
    \begin{equation}\begin{aligned}
        0 \leq |\sigma_{p}(r(A))| \leq |r(A)|
    \end{aligned}\end{equation}

\subsubsection{Esempi}
\textbf{ES1:} selezione dei pazienti con residenza a Torino
    \begin{equation}\begin{aligned}
        \sigma_{Residenza = 'TO'}(pazienti)
    \end{aligned}\end{equation}
    \begin{center}\begin{tabular}{|c|c|c|c|c|} \hline
        \textbf{COD} & \textbf{Cognome} & \textbf{Nome} 
        & \textbf{Residenza} & \textbf{AnnoNascita}\\
        \hline
        A102 & Necchi & Luca & TO & 1950\\
        \hline
        B543 & Missoni & Nadia & TO & 1950\\
        \hline
    \end{tabular}\end{center}
\textbf{ES2:} selezione dei pazienti residenti a Torino o a Vercelli
    \begin{equation}\begin{aligned}
        \sigma_{Residenza = 'TO' \wedge Residenza = 'VC'}(pazienti)
    \end{aligned}\end{equation}
    \begin{center}\begin{tabular}{|c|c|c|c|c|} \hline
        \textbf{COD} & \textbf{Cognome} & \textbf{Nome} 
        & \textbf{Residenza} & \textbf{AnnoNascita}\\ \hline
        A102 & Necchi & Luca & TO & 1950\\ \hline
        B543 & Missoni & Nadia & TO & 1950\\ \hline
        B444 & Missoni & Luigi & VC & 2000\\ \hline
    \end{tabular}\end{center}
\textbf{ES3:} selezione dei pazienti non residenti a Torino
    \begin{equation}\begin{aligned}
        \sigma_{\neg Residenza = 'TO'}(pazienti)
    \end{aligned}\end{equation}
    \begin{center}\begin{tabular}{|c|c|c|c|c|} \hline
        \textbf{COD} & \textbf{Cognome} & \textbf{Nome} 
        & \textbf{Residenza} & \textbf{AnnoNascita}\\ \hline
        B372 & Rossigni & Piero & NO & 1940\\ \hline
        B444 & Missoni & Luigi & VC & 2000\\ \hline
        S555 & Rossetti & Gino & AT & 2010\\ \hline
     \end{tabular}\end{center}
     
\section{Operatore proiezione}
Data una relazione $r(A)$ e un insieme di attributi $A_i, ..., A_k$ tutti appartenenti ad $A$, l'operatore di proiezione
    \begin{equation}\begin{aligned}
        \Pi_{A_i, ..., A_k}(r(A))
    \end{aligned}\end{equation}
produce come risultato una relazione avente:
    \begin{itemize}
        \item{Schema: $\{   A_i, ..., A_k   \}$}
        \item{Istanza: tutte le tuple della relazione argomento, ma solo rispetto ai campi $A_i, ..., A_k$}
    \end{itemize}

\subsection{Cardinalità della proiezione}
A prima vista sembrerebbe che la cardinalità della proiezione sia uguale alla cardinalità della relazione argomento, ovvero
    \begin{equation}\begin{aligned}
        |\Pi_{A_i, ..., A_k}(r(A))| = |r(A)|
    \end{aligned}\end{equation}
Invece la cardinalità della proiezione è data da:
    \begin{equation}\begin{aligned}
        |\Pi_{A_i, ..., A_k}(r(A))| \leq |r(A)|
    \end{aligned}\end{equation}
Per capire il perché, vedere l'ES2.\\
Se, però, gli attributi proiettati $A_i, ..., A_k$ formano una \textbf{superchiave} della relazione argomento, allora:
    \begin{equation}\begin{aligned}
        |\Pi_{A_i, ..., A_k}(r(A))| = |r(A)|
    \end{aligned}\end{equation}

\subsubsection{Esempi}
\textbf{ES1:} proiezione della relazione $pazienti$ sull'attributo $COD$ e sull'attributo $Cognome$:
    \begin{equation}\begin{aligned}
        \Pi_{COD, Cognome}(pazienti)
    \end{aligned}\end{equation}
    \begin{center}\begin{tabular}{|c|c|}\hline
        \textbf{COD} & \textbf{Cognome}\\
        \hline
        A102 & Necchi\\ \hline
        B372 & Rossigni\\ \hline
        B543 & Missoni\\ \hline
        B444 & Missoni\\ \hline
        S555 & Rossetti\\ \hline
     \end{tabular}\end{center}
\textbf{ES2:} proiezione della relazione $pazienti$ sull'attributo $Cognome$:
    \begin{center}\begin{tabular}{|c|}\hline
        \textbf{Cognome}\\ \hline
        Necchi\\ \hline
        Rossigni\\ \hline
        Missoni\\ \hline
        Missoni\\ \hline
        Rossetti\\ \hline
     \end{tabular}\end{center}
Questa non è più una relazione valida nel modello relazionale teorico di Codd (non è più un insieme).\\
Dunque, il risultato è un'istanza di relazione senza ripetizioni:
    \begin{center}\begin{tabular}{|c|}\hline
        \textbf{Cognome}\\ \hline
        Necchi\\ \hline
        Rossigni\\ \hline
        Missoni\\ \hline
        Rossetti\\ \hline
     \end{tabular}\end{center}
    
\section{Composizione di operatori}
L'algebra relazionale è \textbf{composizionale}, ovvero si possono costruire espressioni di algebra relazionale componendo insieme operatori.\\
Bisogna fare attenzione a che l'espressione algebrica sia sintatticamente corretta, verificando che gli operatori siano consistenti e coerenti con gli argomenti

\subsubsection{Esempio}
\textbf{ES1:} Elencare codice e nome dei pazienti residenti a Torino o a Vercelli.
\begin{enumerate}
    \item{Selezione dei pazienti residenti a Torino o Vecelli.
        \begin{equation}\begin{aligned}
            \sigma_{Residenza = 'To' \wedge Residenza = 'VC'}(pazienti)
        \end{aligned}\end{equation}}
    \item{La relazione virtuale prodotta dalla selezione può essere usata come argomento per la proiezione.
        \begin{equation}\begin{aligned}
            \Pi_{COD, Nome}(\sigma_{Residenza = 'To' \wedge Residenza = 'VC'}(pazienti))
        \end{aligned}\end{equation}}
\end{enumerate}
Il risultato è una relazione che ha come schema quello della proiezione, e come istanza solo le tuple che soddisfano il predicato della selezione.\\\\
\textbf{ES2:} La relazione 
    \begin{equation}\begin{aligned}
        \sigma_{Residenza = 'To' \wedge Residenza = 'VC'}(\Pi_{COD; Nome}(pazienti))
    \end{aligned}\end{equation}
diventa sintatticamente scorretta perché la relazione virtuale prodotta da $\Pi_{COD; Nome}(pazienti)$ non contiene l'attributo $Residenza$.

\section{Operatori insiemistici: unione e differenza}
Per definizione, gli operatori insiemistici richiedono che gli schemi delle relazioni argomento siano identici.\\\\
Il risultato dell'operatore insiemistico sulle relazioni argomento $r_1(A)$ e $r_2(A)$ è una relazione che ha come schema lo stesso schema $A$ delle relazioni argomento e come istanza:
    \begin{itemize}
        \item{Unione $r_1 \cup r_2$: l'unione delle tuple di $r_1$ e $r_2$}
        \item{Differenza $r_1 - r_2$:  tutte le tuple di $r_1$ che non sono contenute in $r_2$}
    \end{itemize}

\subsection{Cardinalità dell'unione}
La cardinalità dell'unione $r_1(A) \cup r_2(A)$ è data da:
    \begin{equation}\begin{aligned}
        0 \leq |r_1(A) \cup r_2(A)| \leq |r_1(A)| + |r_2(A)|
    \end{aligned}\end{equation}
La cardinalità può essere minore della somma delle cardinalità delle due relazioni se ci sono delle tuple ripetute.

\subsection{Cardinalità della differenza}
La cardinalità della differenza $r_1(A) - r_2(A)$ è:
    \begin{equation}\begin{aligned}
        0 \leq |r_1(A) - r_2(A)| \leq |r_1(A)|
    \end{aligned}\end{equation}

\subsubsection{Esempi}
\textbf{ES1}: elencare cognomi e nomi di tutte le persone coinvolte nel database "Ricoveri Ospedalieri" (quindi, medici e pazienti):
    \begin{equation}\begin{aligned}
        \Pi_{Cognome, Nome} (pazienti) \cup
        \Pi_{Cognome, Nome} (medici)
    \end{aligned}\end{equation}
    \begin{center}\begin{tabular}{|c|c|}\hline
        \textbf{Cognome} & \textbf{Nome}\\ \hline
        Necchi & Lucca \\ \hline
        Rossigni & Piero\\ \hline
        Missoni & Nadia\\ \hline
        Missoni & Luigi\\ \hline 
        Rossetti & Gino\\ \hline
        Neri & Piero\\ \hline
        Bisi & Mario\\ \hline
        Bargio & Sergio\\ \hline
        Belli & Nicola\\ \hline
        Mizzi & Nicola\\ \hline
        Monti & Mario\\ \hline
    \end{tabular}\end{center}
\textbf{ES2}: elencare le province di residenza dei medici in cui non risiede alcun paziente
    \begin{equation}\begin{aligned}
        \Pi_{Residenza} (medici) -
        \Pi_{Residenza} (pazienti)
    \end{aligned}\end{equation}
    \begin{center}\begin{tabular}{|c|}\hline
        \textbf{Residenza} \\ \hline
        NO \\ \hline
    \end{tabular}\end{center}
    
\section{Operatore intersezione}
Esattamente come gli altri due operatori insiemistici, l'operatore intersezione è definito su relazioni aventi lo stesso schema.\\
Il risultato dell'intersezione $r_1(A) \cap r_2(A)$ è una relazione che ha come schema lo stesso schema $A$ delle relazioni argomento e istanza tutte le tuple di $r_1$ contenute anche in $r_2$.\\\\
L'operatore di intersezione insiemistica può essere reso con l'operatore di differenza insiemistica:
    \begin{equation}\begin{aligned}
        r_1(A) \cap r_2(A) = r_1(A) - (r_1(A) - r_2(A))
    \end{aligned}\end{equation}
    
\subsection{Cardinalità dell'intersezione}
La cardinalità dell'intersezione soddisfa la seguente relazione:
    \begin{equation}\begin{aligned}
        0 \leq |r_1(A) \cap r_2(A)| \leq \min{|r_1(A)|, |r_2(A)|}
    \end{aligned}\end{equation}

\section{Operatore ridenominazione}
Data una relazione $r(A)$ con $A = \{   A_1, ..., A_i, ..., A_k, ... A_n   \}$ il risultato della ridenominazione
    \begin{equation}\begin{aligned}
        \rho_{B_i, ..., B_k \leftarrow
        A_i, ..., A_k}(r)
    \end{aligned}\end{equation}
è una relazione virtuale, che non modifica la relazione della base dati $r'(A')$ avente:
    \begin{itemize}
        \item{Schema:
            \begin{equation}\begin{aligned}
                A' = \{ A_1, ..., A_i, ..., A_k, ..., A_n   \}
            \end{aligned}\end{equation}}
        \item{Istanza:
            \begin{equation}\begin{aligned}
                r' = r (\text{stesse tuple})
            \end{aligned}\end{equation}}
    \end{itemize}

\subsubsection{Esempio}
\textbf{ES1:} Elencare i medici che non sono primari:
    \begin{equation}\begin{aligned}
        \Pi_{MATR} (medici) - 
        \rho_{MATR \leftarrow Primario}(\Pi_{Primario} (reparti))
    \end{aligned}\end{equation}
    
\section{Uso corretto degli operatori insiemistici}
Nell'utilizzo degli operatori insiemistici abbinati all'operatore di ridenominazione, bisogna assicurarsi sempre che gli attributi delle relazioni siano definiti su domini omogenei (coerenti).\\\\
\textbf{ES1:} i seguenti sono insiemi con domini omogenei:
    \begin{equation}\begin{aligned}
        \Pi_{MATR}(medici) - \rho_{MATR \leftarrow Primario}
        (\Pi_{Primario}(reparti))
    \end{aligned}\end{equation}
\textbf{ES2:} i seguenti sono insiemi con domini non omogenei (anche se gli attributi sono dello stesso tipo):
    \begin{equation}\begin{aligned}
        \Pi_{Nome}(medici) - \rho_{Nome \leftarrow Nome-Rep}
        (\Pi_{Nome-Rep}(pazienti))
    \end{aligned}\end{equation}

\section{Ridenominazione schema}
Data una relazione $r$ con schema:
    \begin{equation}\begin{aligned}
        R(A) \qquad A = \{  A_1, ..., A_i, ..., A_k, ..., A_n   \}
    \end{aligned}\end{equation}
Il risultato della ridenominazione dello schema 
    \begin{equation}\begin{aligned}
        \rho_{R'(B_i, ..., B_k) \leftarrow R(A_i, ..., A_k)} (r)
    \end{aligned}\end{equation}
è una relazione virtuale (che non modifica la relazione della base dati) con schema:
    \begin{equation}\begin{aligned}
        R'(A_1, ..., B_i, ..., B_k, ..., A_n)
    \end{aligned}\end{equation}
\textbf{ES:} 
    \begin{equation}\begin{aligned}
        \rho_{UTENTI(CF, Provincia) \leftarrow PAZIENTI(COD, Residenza)}(pazienti)\\
        UTENTI(CF, Cognome, Nome, Provincia, AnnoNascita)
    \end{aligned}\end{equation}

\section{Prodotto cartesiano}
Date due relazioni $r_1(A)$ e $r_2(B)$, con $A \cap B = \emptyset$ (i due schemi non hanno attributi in comune), il prodotto cartesiano:
    \begin{equation}\begin{aligned}
        r_1(A) \times r_2(B)
    \end{aligned}\end{equation}
Produce come risultato una relazione $r'$ avente come risultato:
    \begin{itemize}
        \item{Schema: $R'$ composto dall'unione degli schemi $A \cup B$}
        \item{Istanza: giustapposizione (combinazione) di tutte le tuple di $r_1(A)$ con tutte le tuple di $r_2(B)$}
    \end{itemize}
A differenza della sua controparte matematica, nel modello di Codd il prodotto cartesiano gode della proprietà commutativa:
    \begin{equation}\begin{aligned}
        r_1(A) \times r_2(B) = r_2(A) \times r_2(B)
    \end{aligned}\end{equation}
Infatti, nella relazione di Codd:
    \begin{itemize}
        \item{Nello schema di una relazione non è importante l'ordine degli attributi.}
        \item{Nell'istanza di una relazione non è importante l'ordine delle tuple.}
    \end{itemize}
\subsubsection{Esempi}
\textbf{ES1:}
    \begin{equation}\begin{aligned}
        ABC \times DE
    \end{aligned}\end{equation}
    \begin{center}\begin{tabular}{|c|c|c|}\hline
        \textbf{A} & \textbf{B} & \textbf{C} \\ \hline
        a1 & b1 & c1 \\ \hline
        a2 & b2 & c2 \\ \hline
    \end{tabular}\quad
    \begin{tabular}{|c|c|}\hline
        \textbf{D} & \textbf{E} \\ \hline
        d1 & e1 \\ \hline
        d2 & e2 \\ \hline
        d3 & e3 \\ \hline
    \end{tabular}\quad
    \begin{tabular}{|c|c|c|c|c|}\hline
        \textbf{A} & \textbf{B} & \textbf{C} & 
        \textbf{D}  & \textbf{E} \\ \hline
        a1 & b1 & c1 & d1 & e1\\ \hline
        a1 & b1 & c1 & d2 & e2\\ \hline
        a1 & b1 & c1 & d3 & e3\\ \hline
        a2 & b2 & c2 & d1 & e1\\ \hline
        a2 & b2 & c2 & d2 & e2\\ \hline
        a2 & b2 & c2 & d3 & e3\\ \hline
    \end{tabular}\end{center}
\textbf{ES2:}
    \begin{equation}\begin{aligned}
        DE \times ABC
    \end{aligned}\end{equation}
    \begin{center}\begin{tabular}{|c|c|}\hline
        \textbf{D} & \textbf{E} \\ \hline
        d1 & e1 \\ \hline
        d2 & e2 \\ \hline
        d3 & e3 \\ \hline
    \end{tabular}\quad
    \begin{tabular}{|c|c|c|}\hline
        \textbf{A} & \textbf{B} & \textbf{C} \\ \hline
        a1 & b1 & c1 \\ \hline
        a2 & b2 & c2 \\ \hline
    \end{tabular} \quad
    \begin{tabular}{|c|c|c|c|c|}\hline
        \textbf{D} & \textbf{E} &
        \textbf{A} & \textbf{B} & \textbf{C} \\ \hline
        d1 & e1 & a1 & b1 & c1  \\ \hline
        d1 & e1 & a2 & b2 & c2  \\ \hline
        d2 & e2 & a1 & b1 & c1  \\ \hline
        d2 & e2 & a2 & b2 & c2  \\ \hline
        d3 & e3 & a1 & b1 & c1  \\ \hline
        d3 & e3 & a2 & b2 & c2  \\ \hline
    \end{tabular}\end{center}
    
\subsection{Cardinalità del prodotto cartesiano}
La cardinalità del prodotto cartesiano $r_1(A) \times r_2(B)$ rispetta la seguente relazione:
    \begin{equation}\begin{aligned}
        0 \leq |r_1(A) \times r_2(B)| = |r_1(A)| \cdot |r_2(B)|
    \end{aligned}\end{equation}
Infatti: 
    \begin{itemize}
        \item{$r_1$ e $r_2$ sono insiemi di tuple, quindi le tuple sono tutte disgiunte}
        \item{giustapponendo tuple disgiunte con le altre tuple disgiunte ottengo una relazione di tuple disgiunte}
    \end{itemize}

\section{Operatore $\bowtie_{\theta}$}
L'operatore di $\bowtie_{\theta}$ serve a costruire informazioni estratte da più relazioni e a mettere in correlazione informazioni di una relazione con informazioni di un'altra relazione.\\\\
Prendendo in input:
    \begin{itemize}
        \item{Due relazioni $r_1(A)$ e $r_2(B)$ con $A \cap B = \emptyset$}
        \item{Una condizione (predicato) di join $\theta$ da intendersi come una congiunzione di confronti tra attributi del tipo $A_i \phi B_j$, dove $\phi$ è un operatore di confronto}
    \end{itemize}
Il $\theta$-join è definito come:
    \begin{equation}\begin{aligned}
        r_1(A) \bowtie_{\theta} r_2(B) = 
        \sigma_{\theta} (r_1(A) \times r_2(B))
    \end{aligned}\end{equation}

\subsubsection{Esempi}
\textbf{ES1:} Elencare tutte le informazioni sui pazienti ricoverati.
    \begin{equation}\begin{aligned}
        ricoveri \bowtie_{PAZ = COD} pazienti\\
        \sigma_{PAZ = COD} (ricoverati \times pazienti)
    \end{aligned}\end{equation}
Il prodotto cartesiano produce tutte le combinazioni di tutte le tuple di $ricoveri$ con tutte le tuple di $pazienti$.\\
A questo punto selezioniamo soltanto quelle che soddisfano la condizione di join: $PAZ = COD$.\\\\
\textbf{ES2:} Elencare le informazioni dei primari di ogni reparto.
    \begin{equation}\begin{aligned}
        reparti \bowtie_{Primario = MATR} medici
    \end{aligned}\end{equation}
\textbf{ES3:} Elencare i reparti con le informazioni del primario solo nel caso in cui il primario afferisce al reparto che dirige.
    \begin{equation}\begin{aligned}
        reparti 
        \bowtie_{(Primario = MATR) \wedge (COD = Reparto)} 
        medici
    \end{aligned}\end{equation}
\textbf{ES4:} Elenco dei reparti abbinati ai dati dei rispettivi primari solo nel caso in cui il primario non vi afferisca.
    \begin{equation}\begin{aligned}
        reparti 
        \bowtie_{(Primario = MATR) \wedge (COD \neq Reparto)} 
        medici
    \end{aligned}\end{equation}
    
\subsection{Cardinalità del $\bowtie_{\theta}$}
La cardinalità di del $\bowtie_{\theta}$ si calcola sfruttando la cardinalità della selezione, e soddisfa la seguente relazione:
    \begin{equation}\begin{aligned}
        0 \leq |\sigma_{\theta}(r_1(A) \times r_2(B))| \leq |r_1(A) \times r_2(B)|
    \end{aligned}\end{equation}
Sappiamo, però, che $|r_1(A) \times r_2(B)| = |r_1(A)| \cdot |r_2(B)|$, quindi:
    \begin{equation}\begin{aligned}
        0 \leq |r_1(A) \times r_2(B)| \leq |r_1(A)| \cdot |r_2(B)|
    \end{aligned}\end{equation}
La cardinalità del $\theta$-join ha un range molto ampio.

\subsection{Join con schemi non disgiunti}
Quando effettuiamo il join tra schemi non disgiunti (che hanno quindi degli attributi in comune), dobbiamo rinominare gli attributi che causano ambiguità almeno in uno dei due schemi.\\\\
\textbf{ES1:} elencare i pazienti abbinati ai medici che risiedono nella stessa città.
    \begin{equation}\begin{aligned}
        \rho_{CognomeM, NomeM, ResidenzaM \leftarrow Cognome, Nome, Residenza}(pazienti)\\ 
        \bowtie_{ResidenzaP = ResidenzaM}\\
        \rho_{CognomeP, NomeP, ResidenzaP \leftarrow Cognome, Nome, Residenza}(medici)
    \end{aligned}\end{equation}

\subsubsection{Esempio}
Quali medici hanno avuto in cura il paziente Missoni Luigi?\\\\
    \begin{enumerate}
        \item{Per sapere dov'è stato ricoverato ho bisogno di combinare i dati della relazione pazienti con la relazione ricoveri
            \begin{equation}\begin{aligned}
                \sigma_{Cognome = 'Missoni' \wedge Nome = 'Luigi'} (pazienti) \bowtie_{COD = PAZ} (ricoveri)
            \end{aligned}\end{equation}}
        \item{Per conoscere i medici afferenti al reparto ho bisogno della relazione medici, che però, ha molti attributi omonimi ad attributi della relazione virtuale prodotta al punto precedente}
        \item{Ridenomino gli attributi di medici
            \begin{equation}\begin{aligned}
                \rho_{CM, NM, RM, Rep \leftarrow
                Cognome, Nome, Residenza, Reparto} (medici) 
            \end{aligned}\end{equation}}
        \item{Metto in join le due espressioni:
            \begin{equation}\begin{aligned}
                [\sigma_{Cognome = 'Missoni' \wedge Nome = 'Luigi'} (pazienti) \bowtie_{COD = PAZ} (ricoveri)]\\
                \bowtie_{Reparto = Rep}\\
                \rho_{CM, NM, RM, Rep \leftarrow
                Cognome, Nome, Residenza, Reparto} (medici) 
            \end{aligned}\end{equation}}
        \item{Infine proietto sui dati che mi interessano:
            \begin{equation}\begin{aligned}
                \Pi_{COD, AnnoNascita, CM, NM}\\
                [\sigma_{Cognome = 'Missoni' \wedge Nome = 'Luigi'} (pazienti) \bowtie_{COD = PAZ} (ricoveri)\\
                \bowtie_{Reparto = Rep}\\
                \rho_{CM, NM, RM, Rep \leftarrow
                Cognome, Nome, Residenza, Reparto} (medici)]
            \end{aligned}\end{equation}}
    \end{enumerate}
    
\subsection{Semplificazione della ridenominazione}
L'operatore di ridenominazione appesantisce la lettura delle espressioni algebriche.\\
Quando ho attributi con lo stesso nome su relazioni diverse, posso considerare il nome dell'attributo specificando la tavola a cui appartiente, usando la \textbf{dot-notation}.\\\\
Facendo riferimento all'esempio precedente, l'interrogazione diventa:
    \begin{equation}\begin{aligned}
        \Pi_{COD, AnnoNascita, medici.Cognome, medici.Nome}\\
        [\sigma_{Cognome = 'Missoni' \wedge Nome = 'Luigi'} (pazienti) \bowtie_{COD = PAZ} (ricoveri)]\\
        \bowtie_{Reparto = Rep} (medici) 
    \end{aligned}\end{equation}
    
\section{Operatore Natural-Join}
Date due relazioni $r_1(A)$ e $r_2(B)$, con 
$A = \{\textbf{X}, Y \}$ e 
$B = \{\textbf{X}, Z \}$ (X, Y e Z sono insiemi disgiunti di attributi) e $Y \cap Z = \emptyset$.\\\\
Il natural-join è definito come
    \begin{equation}\begin{aligned}
        r_1(A) \bowtie r_2(B) = \\
        \Pi_{X, Y, Z} (r_1 \bowtie_{\theta} 
        \rho_{X' \leftarrow X}(r_2))
    \end{aligned}\end{equation}
dove $\theta = (X_i = X_i') \wedge ... \wedge (X_k = X_k')$\\\\
\textbf{Informalmente}: \textit{il natural join tra due relazioni che hanno almeno un attributo in comune ha come condizione (sottintesa) l'uguaglianza tra gli attributi che hanno lo stesso nome.}

\section{Operatore Semi-Join}
Date due relazioni $r_1(A)$ e $r_2(B)$ definiamo il semi-join come:
    \begin{equation}\begin{aligned}
        r_1(A) \ltimes_{\theta} r_2(B) = \\
        \Pi_{A} (r_1(A) \bowtie{\theta} r_2(B))
    \end{aligned}\end{equation}
L'operatore di semi-join funziona da selettore sulla prima tabella sfruttando la seconda.

\subsubsection{Esempi}
\textbf{ES1:} elencare tutti i dati dei primari.\\
Per risolvere l'interrogazione selezioniamo solo i medici che sono anche primari:
    \begin{equation}\begin{aligned}
        medici \ltimes_{MATR = Primario} reparti
    \end{aligned}\end{equation}
\textbf{ES2:} elencare tutti i dati dei pazienti ricoverati in chirurgia:
    \begin{equation}\begin{aligned}
        pazienti \ltimes_{COD = PAZ}\\
        (ricoveri \ltimes_{Reparto = reparti.COD}\\
        (\sigma_{Nome-Rep = 'Chirurgia'}(reparti)))
    \end{aligned}\end{equation}
    
\subsection{Cardinalità del semi-join}
La cardinalità del semi-join è uguale alla cardinalità della proiezione sullo schema $A$, quindi al massimo alla cardinalità della relazione costruita su $A$
    \begin{equation}\begin{aligned}
        |r_1(A) \ltimes_{\theta} r_2(B)| = 
        |\Pi_A((r_1(A) \bowtie_{\theta} r_2(B))|\\
        0 \leq |\Pi_A((r_1(A) \bowtie_{\theta} r_2(B))| \leq |r_1(A)|\\
        0 \leq |r_1(A) \ltimes_{\theta} r_2(B)| \leq |r_1(A)|
    \end{aligned}\end{equation}
    
\section{Join esterni}
Chiamiamo $L$ la tabella a sinistra del join e $R$ la tabella a destra.

\subsection{Left Join}
Il left join viene indicato con $L \leftouterjoin R$, ed consiste nel seguente procedimento:
    \begin{enumerate}
        \item{Vengono prese in considerazione tutte le tuple della tavola $L$, nessuna escluse.}
        \item{Si verifica se le tuple della tabella $L$ sono in join con tuple della tabella $R$.}
        \item{Qualora una tupla di $L$ non faccia join con nessuna tupla della tavola $R$, si inseriscono valori nulli in corrispondenza degli attributi della seconda tavola.}
    \end{enumerate}

\subsection{Right Join}
Il left join viene indicato con $L \rightouterjoin R$, ed consiste nel seguente procedimento:
    \begin{enumerate}
        \item{Vengono prese in considerazione tutte le tuple della tavola $R$, nessuna escluse.}
        \item{Si verifica se le tuple della tabella $R$ sono in join con tuple della tabella $L$.}
        \item{Qualora una tupla di agenzie non faccia join con nessuna tupla della tavola $L$, si inseriscono valori nulli in corrispondenza degli attributi della prima tavola.}
    \end{enumerate}

\subsection{Full Join}
Il full join è l'unione del left join e del right join:
    \begin{equation}\begin{aligned}
        L \fullouterjoin R = (L \leftouterjoin R) \cup (L \rightouterjoin R)
    \end{aligned}\end{equation}

\chapter{Tipologie di interrogazioni}
\section{Interrogazioni con negazione}
La negazione all'interno di un'interrogazione può essere di due tipi: essenziale o non essenziale, a seconda della possibilità o meno di riscrivere l'interrogazione senza usare la negazione.\\\\
\textbf{ES1:} l'interrogazione \textit{"elencare i pazienti non residenti a Torino"} può essere riformulata come \textit{"elencare i pazienti con residenza diversa da Torino"}. Questa è interrogazione con negazione non essenziale.\\\\
\textbf{ES2:} l'interrogazione \textit{"elencare i medici non primari"} non può essere riformulata senza utilizzare la negazione. Quindi, questa interrogazione contiene una negazione non essenziale.

\subsection{Negazione esssenziale}
Lo schema generale di risoluzione di un'interrogazione con negazione essenziale consta dei passi seguenti:
    \begin{enumerate}
        \item{Definire l'universo del discorso $U$}
        \item{Rispondere alla domanda in forma positiva $P$}
        \item{Trovo la risposta $R$ all'interrogazione usando l'operatore differenza: $R = U-P$}
    \end{enumerate}
Ovviamente, per poter utilizzare la differenza, gli schemi di $U$ e $P$ devono essere uguali (e, ovviamente, compatibili).

\subsection{Esempi}
\textbf{ES1:} elencare i medici non primari
    \begin{equation}\begin{aligned}
        U = medici \\
        P = \Pi_{MATR, Cognome, Nome, Residenza, Reparto} (medici \bowtie_{MATR = Primario} reparti)\\
        R  = medici - \Pi_{MATR, Cognome, Nome, Residenza, Reparto} (medici \bowtie_{MATR = Primario} reparti)
    \end{aligned}\end{equation}
\textbf{ES2:}
    \begin{equation}\begin{aligned}
        U = \Pi_{COD, Cognome, Nome, Residenza}(pazienti)\\\\
        P = \Pi_{COD, pazienti.Cognome, pazienti.Nome,      pazienti.Residenza} \\
        (pazienti \bowtie_{pazienti.Residenza = medici.Residenza} medici)\\\\
        R = \Pi_{COD, Cognome, Nome, Residenza} (pazienti) -\\
        \Pi_{COD, pazienti.Cognome, pazienti.Nome,     pazienti.Residenza}\\
        (pazienti \bowtie_{pazienti.Residenza = medici.Residenza} medici)
    \end{aligned}\end{equation}

\section{Interrogazioni con quantificazione universale}
Tutti gli esempi visti finora ricadono nella casistica della quantificazione esistenziale.\\\\
\textbf{Esiste un medico non primario}\\
\textbf{Esiste un paziente non residente a Torino}\\
Quando, invece troviamo le parole \textbf{"tutti", "ogni", "sempre"} nell'interrogazione, ci troviamo davanti a casi di quantificazione universale.\\
Possiamo risolvere questo tipo di interrogazioni riconducendoci alla quantificazione esistenziale, ricordandoci che la quantificazione universale è la negazione di una quantificazione esistenziale.

\subsection{Operatore deerivato quoziente}
Dati due insiemi disgiunti di attributi $A$ e $B$, l'operatore derivato quoziente si definisce come:
    \begin{equation}\begin{aligned}
        r(A, B) \div s(B) = \Pi_A (r) - \Pi_A ((\Pi_A(r) \times s) - r)
    \end{aligned}\end{equation}
\subsubsection{Esempio1}

Consideriamo le seguenti tabelle $E$ = esami, e $P$ = piano di studi.
    \begin{center}\begin{tabular}{|c|c|}\hline
        \textbf{MATR} & \textbf{Corso} \\ \hline
        2 & Programmazione \\ \hline
        3 & Algebra \\ \hline
        2 & Basi di dati \\ \hline
        3 & Programmazione \\ \hline
        2 & Algebra \\ \hline
    \end{tabular} \quad \begin{tabular}{|c|}\hline
        \textbf{Corso}\\ \hline
        Programmazione\\ \hline
        Basi di dati\\ \hline
        Algebra\\ \hline
    \end{tabular}\end{center}
Elencare le matricole abbinate a tutti i corsi previsti dal piano di studi.\\\\
\textbf{1}: Individuo l'universo di riferimento
    \begin{equation}\begin{aligned}
        U = \Pi_{MATR} (e)
    \end{aligned}\end{equation}
\textbf{2}: Analizzo tutte le combinazioni possibili di matricola con i corsi previsti:
    \begin{equation}\begin{aligned}
        \Pi_{MATR}(e) \times p
    \end{aligned}\end{equation}
    \begin{center}\begin{tabular}{|c|c|}\hline
        \textbf{MATR} & \textbf{Corso} \\ \hline
        2 & Programmazione \\ \hline
        2 & Basi di dati \\ \hline
        2 & Algebra \\ \hline
        3 & Programmazione \\ \hline
        3 & Basi di dati \\ \hline
        3 & Algebra \\ \hline
    \end{tabular}\end{center}
\textbf{3}: Trovo la differenza con $e$ (hanno lo stesso schema)
    \begin{equation}\begin{aligned}
        (\Pi_{MATR}(e) \times p) - e
    \end{aligned}\end{equation}
    \begin{center}\begin{tabular}{|c|c|} \hline
        \textbf{MATR} & \textbf{Corso} \\ \hline
        3 & Basi di dati \\ \hline
    \end{tabular}\end{center}
\textbf{4}: Proiettando questo risultato su $MATR$, otteniamo l'elenco delle matricole che non hanno ancora superato qualche corso.
    \begin{equation}\begin{aligned}
        \Pi_{MATR}((\Pi_{MATR}(e) \times p) - e)
    \end{aligned}\end{equation}
\textbf{5}: A questo punto possiamo riconsiderare l'universo di partenza, e sottrarre ad esso il risultato ottenuto: 
    \begin{equation}\begin{aligned}
        \Pi_{MATR}(e) - \Pi_{MATR}((\Pi_{MATR}(e) \times p) - e)
    \end{aligned}\end{equation}
    
\subsubsection{Esempio 2}
Elencare i pazienti che hanno subito ricovero in ogni reparto.\\\\
Espandiamo l'operatore quoziente:
    \begin{equation}\begin{aligned}
        \Pi_{PAZ, Reparto}(ricoveri) \div \Pi_{Reparto}(ricoveri) = \\\\
        \Pi_{PAZ}(\Pi_{PAZ, Reparto}(ricoveri)) -\\
        \Pi_{PAZ}((\Pi_{PAZ}(\Pi_{PAZ, Reparto}(ricoveri)) 
        \times \Pi_{Reparto}(ricoveri)) - \\
        \Pi_{PAZ, Reparto}(ricoveri))
    \end{aligned}\end{equation}

\section{Semantica di Codd del valore nullo}
Fin qui abbiamo ipotizzato l'assenza di valori nulli.\\
In realtà, però, il modello di Codd ne prevede l'esistenza. Il valore nullo è un valore vero e proprio, che appartiene a ogni dominio.\\\\
Il valore nullo può avere differenti significati:
    \begin{itemize}
        \item{Informazione esistente, ma mancante.}
        \item{Informazione inesistente}
        \item{Assenza di informazione}
    \end{itemize}
La semantica di Codd del valore nullo è il comportamento frontale di una interrogazione a fronte di tuple che contengono valori nulli.\\\\
In particolare, ci riferiamo alla selezione $\sigma_p$ con predicato $p$, che è una combinazione di confronti.\\
Il problema che si pone è: che valore di verità assume il confronto quando una tupla ha valore nullo in corrispondenza dell'attributo oggetto del confronto?\\
Quando il valore della tupla in corrispondenza dell'attributo $A_i$ è nullo, il valore di verità del confronto di un attributo con la costante è \textbf{sconosciuto}.

\section{Logica a tre valori}
Codd propone di collocarsi in una logica a tre valori: $Falso(F), Vero(V), Sconosciuto(U)$.\\
Quindi, se $t.A_i$ è nullo, il valore di verità del confronto   
    \begin{equation}\begin{aligned}
        t.A_i \theta \text{costante}
    \end{aligned}\end{equation}
sarà $U$.

\subsection{Operatori e logica a tre valori}
Poiché i predicati $\theta$ sono spesso formati da una composizione di AND, OR, NOT, vediamo le \textbf{tavole di verità} dei tre operatori nella logica a tre valori:
    \begin{center}
        \begin{tabular}{|c|c|c|c|}\hline
            \textbf{AND} & \textbf{F} & \textbf{U} & \textbf{T} \\ \hline
            \textbf{F} & F & F & F \\ \hline
            \textbf{U} & F & U & U \\ \hline
            \textbf{T} & F & U & T \\ \hline
        \end{tabular} \quad
        \begin{tabular}{|c|c|c|c|}\hline
            \textbf{OR} & \textbf{F} & \textbf{U} & \textbf{T} \\ \hline
            \textbf{F} & F & U & T \\ \hline
            \textbf{U} & U & U & T \\ \hline
            \textbf{T} & T & T & T \\ \hline
        \end{tabular} \quad
        \begin{tabular}{|c|c|}\hline
            \textbf{NOT} & \\ \hline
            \textbf{F} & T \\ \hline
            \textbf{U} & U \\ \hline
            \textbf{T} & F \\ \hline
        \end{tabular} 
    \end{center}

\subsection{Espressioni e logica a tre valori}
Date le tabelle di verità a tre valori, ogni tupla presa in considerazione genera un valore di verità dell'intera espressione che può essere F, T o U.\\
Ma quando eseguo una selezione e trovo una tupla che ha valore di verità sconosciuto, seleziono o no la tupla?\\\\
Considerando una tupla $t$, per sapere se questa è selezionata o meno, usiamo la \textbf{funzione di collasso} del predicato $p(t)$:   
    \begin{center}\begin{tabular}{|c|c|}\hline
        \textbf{p(t)} & \textbf{c(p(t))} \\ \hline
        F & F \\ \hline
        U & F \\ \hline
        T & T \\ \hline
    \end{tabular}\end{center}
La tupla viene considerata solo se la funzione di collasso è T.\\\\
Tutte le volte che una tupla presenta valore nullo, in corrispondenza della selezione dà sempre valore di verità sconosciuto.\\
Per poter cercare una tupla che ha (o non ha) un valore nullo in corrispondenza dell'attributo $A_i$ scrivo un predicato con le clausole:
    \begin{equation}\begin{aligned}
        A_i \texttt{ IS NULL} \quad\text{ se mi interessa il valore nullo}\\
        A_i \texttt{ IS NOT NULL} \quad \text{ se mi interessano i valori non nulli}
    \end{aligned}\end{equation}
\textbf{ES1:} 
    \begin{equation}\begin{aligned}
        \sigma_{Residenza \texttt{ IS NULL}} (clienti)
    \end{aligned}\end{equation}
    \begin{center}\begin{tabular}{|c|c|} \hline
        \textbf{COD} & \textbf{Residenza} \\ \hline
        200 & NULL \\ \hline
        400 & NULL \\ \hline
    \end{tabular}\end{center}
\textbf{ES2:}
    \begin{equation}\begin{aligned}
        \sigma_{Residenza \texttt{ IS NOT NULL}} (clienti)
    \end{aligned}\end{equation}
    \begin{center}\begin{tabular}{|c|c|} \hline
        \textbf{COD} & \textbf{Residenza} \\ \hline
        100 & TO \\ \hline
        300 & TO \\ \hline
    \end{tabular}\end{center}

\subsection{Limiti della logica a tre valori}
La logica a tre valori viola il \textbf{principio di non contraddizione} per cui vale:
    \begin{equation}\begin{aligned}
        (p \wedge \neg p) = F 
    \end{aligned}\end{equation}
Se però, $p = U$, abbiamo 
    \begin{equation}\begin{aligned}
        (p \wedge \neg p) = U
    \end{aligned}\end{equation}
La logica a tre valori, inoltre, viola il \textbf{principio del terzo escluso}:
    \begin{equation}\begin{aligned}
        (p \vee \neg p) = T
    \end{aligned}\end{equation}
Se però, $p = U$, abbiamo 
    \begin{equation}\begin{aligned}
        (p \vee \neg p) = U
    \end{aligned}\end{equation}

\subsubsection{Esempio}
Ipotizziamo che $pazienti$ ammetta valore nullo in corrispondenza dell'attributo $AnnoNascita$, la selezione
    \begin{equation}\begin{aligned}
        \sigma_{AnnoNascita < 1980 \vee AnnoNascita \geq 1980} (pazienti)
    \end{aligned}\end{equation}
contiene un predicato della forma $(p \vee \neg p)$, ma la funzione collasso dà falso, per cui:
    \begin{equation}\begin{aligned}
        |\sigma_{AnnoNascita < 1980 \vee AnnoNascita \geq 1980} (pazienti)| \leq|pazienti|
    \end{aligned}\end{equation}
L'interrogazione 
    \begin{equation}\begin{aligned}
        \sigma_{AnnoNascita < 1980 \vee AnnoNascita \geq 1980} (pazienti)
    \end{aligned}\end{equation}
è equivalente a 
    \begin{equation}\begin{aligned}
        \sigma_{AnnoNascita \texttt{ IS NOT NULL}} (pazienti)
    \end{aligned}\end{equation}

\subsection{Perché la logica a tre valori}
Immaginiamo un confronto $\neg(A_i = \text{ costante})$:
    \begin{itemize}
        \item{Quando arriva una tupla che ha valore nullo in corrispondenza di $A_i$ l'uguaglianza con la costante genera sconosciuto}
        \item{La funzione \texttt{NOT} applicata ad $A_i$ genera sconosciuto}
        \item{La funzione collasso non fa selezionare la tupla}
    \end{itemize}
Se il confronto tra un valore nullo e una costante avesse assunto il valore F, la negazione avrebbe resituito il valore T!

\section{Proprietà degli operatori algebrici}
Le proprietà degli operatori ci permettono di semplificare espressioni complesse o per determinare l'equivalenza tra due espressioni.\\
Inoltre, le proprietà sono effettivamente utilizzate dai DBMS per ottimizzare le interrogazioni.

\subsection{Proprietà commutativa del prodotto cartesiano}
Ricordando che lo schema risultato del prodotto cartesiano è l'unione dello schema, e che l'ordine degli attributi in una relazione di Codd non è rilevante, possiamo concludere che il prodotto cartesiano è commutativo, cioè:
    \begin{equation}\begin{aligned}
        r(A) \times s(B) = s(B) \times r(A)
    \end{aligned}\end{equation}
    
\subsection{Proprietà commutativa del $\theta$-join}
Ricordando che il prodotto cartesiano è commutativo e che possiamo esprimere il $\theta$-join come una selezione sul prodotto cartesiano, concludiamo che il $\theta$-join è commutativo:
    \begin{equation}\begin{aligned}
        r(A) \bowtie_{\theta} s(B) = s(B) \bowtie_{\theta} r(A)
    \end{aligned}\end{equation}

\subsection{Proprietà associativa del prodotto cartesiano}
Consideriamo le due espressioni :
    \begin{equation}\begin{aligned}
        (r(A) \times s(B)) \times u(C)\\
        r(A) \times (s(B) \times u(C))
    \end{aligned}\end{equation}
Ricordando che:
    \begin{itemize}
        \item{Lo schema di $r(A) \times s(B)$ è $A \cup B$}
        \item{Lo schema di $(r(A) \times s(B)) \times u(C)$ è $(A \cup B) \cup C$}
        \item{Lo schema di $(A \cup B) \cup C)$ è uguale a $A \cup (B \cup C)$}
    \end{itemize}
Possiamo concludere, di conseguenza, che i due schemi coincidono, e che, quindi il prodotto cartesiano è associativo:
    \begin{equation}\begin{aligned}
        (r(A) \times s(B)) \times u(C) = r(A) \times (s(B) \times u(C))
    \end{aligned}\end{equation}

\subsection{Proprietà associativa del $\theta$-join}
Ricordando che il prodotto cartesiano è associativo e che possiamo esprimere il $\theta$-join come una selezione sul prodotto cartesiano, concludiamo che il $\theta$-join è associativo:
    \begin{equation}\begin{aligned}
        (r(A) \bowtie_{\theta_1} s(B))  \bowtie_{\theta_2} u(C) = 
        r(A) \bowtie_{\theta_1} (s(B)  \bowtie_{\theta_2} u(C))
    \end{aligned}\end{equation}
    
\subsection{Proprietà della selezione multipla}
    \begin{equation}\begin{aligned}
        \sigma_{p \wedge q} (r(A))= \sigma_p(\sigma_q(r(A)))
    \end{aligned}\end{equation}
La selezione $p \wedge q$ sceglie le tuple che soddisfano sia il predicato $p$ che il predicato $q$.\\
La selezione di $p$ applicata alle tuple selezionate da $q$, seleziona le tuple che soddisfano contemporanamente $p$ e $q$.

\subsection{Proprietà della sostituzione di operatori}
    \begin{equation}\begin{aligned}
        \sigma_{p \wedge q} (r(A))(r(A))= 
        \sigma_p (r(A)) \cap \sigma_q (r(A))
    \end{aligned}\end{equation}
    \begin{equation}\begin{aligned}
        \sigma_{p \vee q} (r(A))= 
        \sigma_p (r(A)) \cup \sigma_q (r(A))
    \end{aligned}\end{equation}
    \begin{equation}\begin{aligned}
        \sigma_{p \wedge \neg q} (r(A))= 
        \sigma_p (r(A)) - \sigma_q (r(A))
    \end{aligned}\end{equation}
    
\subsection{Proprietà distributiva della selezione rispetto alla proiezione}
    \begin{equation}\begin{aligned}
        \sigma_p \Pi_X (r(A)) = \Pi_X \sigma_p (r(A))
    \end{aligned}\end{equation}
Vale la condizione per cui il predicato $p$ è definito solo su attributi di $X$.\\\\
Notiamo che, se troviamo $\sigma_p \Pi_X (r(A))$ possiamo sempre riscrivere $\Pi_X \sigma_p (r(A))$, mentre il viceversa non è sempre valido.\\
Se troviamo $\Pi_X \sigma_p (r(A))$, per poter scrivere $\sigma_p \Pi_X (r(A))$ è necessario che il predicato $p$ sia definito solo su attributi $X$.\\\\
\textbf{ES1}: l'espressione
    \begin{equation}\begin{aligned}
        \Pi_{COD, AnnoNascita} \sigma_{AnnoNascita > 1980} (pazienti)
    \end{aligned}\end{equation}
Può essere riscritta come
    \begin{equation}\begin{aligned}
        \sigma_{AnnoNascita > 1980} \Pi_{COD, AnnoNascita} (pazienti)
    \end{aligned}\end{equation}
\textbf{ES2:} l'espressione
    \begin{equation}\begin{aligned}
        \Pi_{COD} \sigma_{AnnoNascita > 1980} (pazienti)
    \end{aligned}\end{equation}
Non può essere riscritta come
    \begin{equation}\begin{aligned}
        \sigma_{AnnoNascita > 1980} \Pi_{COD} (pazienti)
    \end{aligned}\end{equation}
    
\subsection{Proprietà distributiva della selezione rispetto al prodotto cartesiano}
    \begin{equation}\begin{aligned}
        \sigma_p(r(A) \times s(B))
    \end{aligned}\end{equation}
Quando $p$ coinvolge sia attributi di $A$ che attributi di $B$ non c'è nessuna possibilità di applicare proprietà distributive.\\\\
Se invece $p$ coinvolge solo attributi contenuti nello schema di una delle due relazioni (ad esempio $A$) è possibile scrivere:
    \begin{equation}\begin{aligned}
        \sigma_p(r(A)) \times s(B)
    \end{aligned}\end{equation}
Una volta applicato il prodotto cartesiano, applichiamo la selezione (per ipotesi) solo alle parti $A$ delle giustapposizioni.\\
Selezionare le tuple giustapposte prendendo solo quelle che soddisfano $p$ è equivalente a selezionare le tuple da $r$ che soddisfano $p$ e giustapporle in seguito alle tuple di $s$.

\subsection{Proprietà distributiva della selezione rispetto al join}
Ricordando che il prodotto cartesiano che possiamo esprimere il $\theta$-join come una selezione sul prodotto cartesiano, concludiamo che possiamo utilizzare la proprietà distributiva della selezione rispetto al join.
    \begin{equation}\begin{aligned}
        \sigma_p(r(A) \bowtie_{\theta} s(B)) = \sigma_p(r(A)) \bowtie_{\theta} s(B)
    \end{aligned}\end{equation}

\subsection{Proprietà distributiva della selezione rispetto all'unione e differenza}
    \begin{equation}\begin{aligned}
        \sigma_p (r(A) \cup s(A)) = \sigma_p (r(A)) \cup \sigma_p (s(A))
    \end{aligned}\end{equation}
    \begin{equation}\begin{aligned}
        \sigma_p (r(A) - s(A)) = \sigma_p (r(A)) - \sigma_p (s(A))
    \end{aligned}\end{equation}
    \begin{equation}\begin{aligned}
        \sigma_p (r(A) \cap s(A)) = \sigma_p (r(A)) \cap \sigma_p (s(A))
    \end{aligned}\end{equation}

\subsection{Proprietà della proiezione multipla}
Supponiamo $X, Y$ sottoinsiemi di $A$:
    \begin{equation}\begin{aligned}
        \Pi_X \Pi_{X, Y} r(A) = \Pi_X r(A)
    \end{aligned}\end{equation}

\subsection{Proprietà distributiva della proiezione rispetto al prodotto cartesiano}
Siano $X_A \subseteq A$ e $X_B \subseteq B$, ovvero:
    \begin{equation}\begin{aligned}
        X_A = X \cap A\\
        X_B = X \cap B
    \end{aligned}\end{equation}
Vale la proprietà distributiva della proiezione rispetto al prodotto cartesiano, cioè:
    \begin{equation}\begin{aligned}
        \Pi_X (r(A) \times s(B)) = \Pi_{X_A} (r(A)) \times \Pi_{X_B} (r(B))
    \end{aligned}\end{equation}

\subsection{Proprietà distributiva della proiezione rispetto al join}
    \begin{equation}\begin{aligned}
        \Pi_X (r(A) \bowtie_{\theta} s(B)) = \Pi_{X_A} (r(A)) \bowtie_{\theta} \Pi_{X_B} (r(B))
    \end{aligned}\end{equation}
Siano $X_A \subseteq A$ e $X_B \subseteq B$, ovvero:
    \begin{equation}\begin{aligned}
        X_A = X \cap A\\
        X_B = X \cap B
    \end{aligned}\end{equation}
Inoltre, è necessario che gli attributi coinvolti nel predicato $\theta$ siano tutti contenuti in $X_A \cup X_B$

\subsection{Proprietà distributiva della proiezione rispetto all'unione}
    \begin{equation}\begin{aligned}
        \Pi_X (r(A) \cup s(A)) = \Pi_X (r(A)) \cup \Pi_X (s(B))
    \end{aligned}\end{equation}

\subsection{Altre proprietà}
Le proprietà distributive della proiezione rispetto alla differenza e all'intersezione non esistono.