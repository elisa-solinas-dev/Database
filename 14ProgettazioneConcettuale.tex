\chapter{Progettazione Concettuale}



\section{Raccolta e analisi dei requisiti}
Per \textbf{raccolta dei requisiti} si intende la completa individuazione dei problemi che l'applicazione da realizzare deve risolvere e le caratteristiche che tale applicazione dovrà avere.\\ 
Per caratteristiche del sistema si intendono sia gli aspetti statici (i dati) che gli aspetti dinamici (le operazioni sui dati).\\
Dato che i requisiti vengono inizialmente raccolti in specifiche espresse generalmente in linguaggio naturale, risultano spesso ambigui e disorganizzati.\\\\
L'\textbf{analisi dei requisiti} consiste nel chiarimento e nell'organizzazione delle specifiche dei requisiti.

\subsection{Fonti dei requisiti}
Per la raccolta dei requisiti possiamo rivolgerci a diverse fonti, che rientrano solitamente in una delle seguenti categorie:
    \begin{itemize}
        \item{\textbf{Utenti dell'applicazione}: le operazioni si acquisiscono mediante opportune interviste, oppure attraverso una documentazione scritta che gli utenti possono avere predisposto appositamente.}
        \item{\textbf{Documentazione esistente}: documentazione che ha qualche attinenza con il problema che stiamo studiando (moduli, regolamenti interni, procedure aziendali, normative). \\
        È richiesta, in questo caso, un'attività di raccolta e selezione che viene assistita dagli utenti, ma è sempre a carico del progettista.}
        \item{\textbf{Realizzazioni pre-esistenti}: applicazioni che devono essere rimpiazzate oppure che devono interagire in qualche maniera con il sistema da realizzare.\\
        La conoscenza delle caratteristiche di queste applicazioni potrebbe fornirci importanti informazioni anche in relazione ai problemi esistenti che è necessario risolvere.}
    \end{itemize}

\subsection{Specifica dei requisiti}
La specifica dei requisiti raccolti avviene spesso facendo uso di descrizioni in linguaggio naturale. Sappiamo, però, che il linguaggio naturale è fonte di ambiguità e fraintendimenti.\\
È molto importante, quindi, effettuare una profonda analisi del testo che descrive le specifiche per filtrare le eventuali inesattezze e i termini ambigui presenti.\\\\
Ecco alcune regole generali per ottenere una specifica dei requisiti più precisa e senza ambiguità:
    \begin{enumerate}
        \item{\textbf{Scegliere il corretto livello di astrazione}: è bene evitare di utilizzare termini \textit{troppo generici} o \textit{troppo specifici}, che rendono poco chiaro un concetto.}
        \item{\textbf{Standardizzare la struttura delle frasi}: nella specifica dei requisiti preferibile utilizzare sempre lo stesso stile sintattico.\\
        Per esempio:
            \begin{equation}
                \text{Per} \quad <dato> \qquad
                \text{rappresentiamo} \quad <insieme \text{ } di \text{ } proprieta>
            \end{equation}}
        \item{\textbf{Evitare frasi contorte}: le definizioni devono essere semplici e chiare.}
        \item{\textbf{Individuare sinonimi/omonimi e unificare i termini}: i sinonimi indicano termini diversi con lo stesso significato; gli omonimi indicano termini uguali con diversi significati.\\
        Queste situazioni possono generare ambiguità e vanno chiarite: nel caso di sinonimi unificando i termini, nel caso di omonimi utilizzando termini diversi o specificandoli meglio.}
        \item{\textbf{Rendere esplicito il riferimento tra termini}: può succedere che l'assenza di di un contesto di riferimento renda alcuni concetti ambigui: in questi casi bisogna esplicitare il riferimento tra termini.}
        \item{\textbf{Costruire un glossario dei termini}: è molto utile per la comprensione e la precisazione dei termini usati, definire un glossario che, per ogni termine, contenga: una breve descrizione, possibili sinonimi e altri termini contenuti nel glossario con i quali esiste un legame logico.}
    \end{enumerate}

    \begin{table}[h!]
        \begin{tabularx}{\textwidth}{|c|X|c|c|} \hline
            \textbf{Termine} & \textbf{Descrizione} & \textbf{Sinonimi} & \textbf{Collegamenti} \\ \hline
            ... & ... & ... & ... \\ \hline
        \end{tabularx}
        \caption{Glossario dei termini}
    \end{table}

\subsection{Specifiche sulle operazioni}
Accanto alle specifiche sui dati, vanno raccolte le specifiche sulle operazioni da effettuare su tali dati.
È opportuno cercare di impiegare la stessa terminologia usata per i dati (possiamo fare riferimento al glossario dei termini) e informarci anche sulla frequenza con la quale le varie operazioni vengono eseguite.



\section{Criteri per la rappresentazione dei dati}

\subsection{Criteri generali di rappresentazione}
Pur specificando che non esiste una rappresentazione univoca di un insieme di specifiche, è utile avere delle indicazioni sulle scelte più opportune.
    \begin{enumerate}
        \item{Se un concetto ha proprietà significative e/o descrive classi di oggetti con esistenza autonoma, è opportuno rappresentarlo con una \textbf{entità}.}
        \item{Se un concetto ha una struttura semplice e non possiede proprietà rilevanti associate, è opportuno rappresentarlo con un attributo di un altro concetto a cui si riferisce.}
        \item{Se sono state individuate due (o più) entità e nei requisiti compare un concetto che le associa, questo concetto può essere rappresentato da una relazione.}
        \item{Se uno o più concetti risultano essere casi particolari di un altro, è opportuno rappresentarli facendo uso di una generalizzazione.}
    \end{enumerate}

\subsection{Qualità di uno schema concettuale}
Nella costruzione di uno schema concettuale vanno garantite alcune proprietà generali che uno schema concettuale di buona qualità deve possedere.

\subsubsection{Correttezza}
Uno schema concettuale è \textbf{corretto} quando utilizza propriamente i costrutti messi a disposizione dal modello concettuale di riferimento.\\
Come avviene nei linguaggi di programmazione, gli errori possono essere di due tipi:
    \begin{itemize}
        \item{\textbf{Errori sintattici}: riguardano un uso non ammesso di costrutti.\\
        E.g. una generalizzazione tra relazioni invece che tra entità.}
        \item{{Errori semantici}: riguardano un uso di costrutti che non rispetta la loro definizione.\\
        E.g. l'uso di una relazione per descrivere il fatto che un'entità è una specializzazione di un'altra}
    \end{itemize}
    
\subsubsection{Completezza}
Uno schema concettuale è \textbf{completo} quando rappresenta tutti i dati di interesse e quando tutte le operazioni possono essere eseguite a partire dai concetti descritti nello schema.

\subsubsection{Leggibilità}
Uno schema concettuale è \textbf{leggibile} quando rappresenta i requisiti in maniera naturale e facilmente comprensibile.\\
Per garantire questa proprietà è necessario rendere lo schema autoesplicativo, per esempio, mediante una scelta opportuna dei nomi da dare ai concetti.\\
Inoltre, la leggibilità dipende anche da fattori puramente estetici. Ecco alcuni suggerimenti per rendere lo schema più leggibile:
    \begin{itemize}
        \item{Disporre i costrutti su una griglia scegliendo come elementi centrali quelli con più legami (relazioni) con altri.}
        \item{Tracciare solo linee perpendicolari e cercare di minimizzare le intersezioni.}
        \item{Disporre le entità che sono generalizzazioni sopra le relative entità figlie.}
    \end{itemize}
    
\subsubsection{Minimalità}
Uno schema è \textbf{minimale} quando tutte le specifiche sui dati sono rappresentate una sola volta nello schema. Uno schema, quindi, non è minimale quando esistono delle \textbf{ridondanze}, ovvero concetti che possono essere derivati da altri.\\
A differenza delle altre proprietà, comunque, non sempre una ridondanza è indesiderata, ma può nascere da precise scelte di progettazione.