\chapter{Calcolo relazionale}

L'algebra relazionale è un linguaggio di tipo \textbf{procedurale}: l'utente indica le operazioni da compiere per arrivare al risultato e il DBMS sceglie poi la strategia ottimale.\\
Il calcolo relazionale, invece, segue un approccio \textbf{dichiarativo}, fondato sulla logica: esprime infatti le caratteristiche di risultato, senza specificare il procedimento utilizzato per ottenerlo.

\section{Calcolo su tuple con dichiarazione di range}
L'interrogazione è composta da tre parti:  
\begin{equation}\begin{aligned}
    \{  T|L|F|  \}
\end{aligned}\end{equation}
\begin{itemize}
    \item{$T$ (\textbf{Target}): la lista di informazioni che voglio in uscita. Possiamo usare una delle seguenti sintassi:
        \begin{equation}\begin{aligned}
            variabile.Attributo1,\text{ } variabile.Attributo2, ...\\
            variabile.(Attributo1,\text{ } Attributo2)\\
            variabile.* \text{ (visualizza tutti gli attributi)}\\
            Nome: \text{ } variabile.Attributo
        \end{aligned}\end{equation}
    Tutte le variabili usate nella Target List devono essere dichiarate nella Range List.}
    \item{$L$ (\textbf{Range List}): introduzione di variabili abbinate a tavole di base, con la seguente sintassi:
        \begin{equation}\begin{aligned}
            nomeVariabile (NomeTavola)
        \end{aligned}\end{equation}}
    \item{$F$(\textbf{Formula}): predicato del primo ordine su predicati di base basati su confronti del tipo:
        \begin{equation}\begin{aligned}
            x.A \text{ }\theta \text{ costante}\\
            x.A_i \text{ }\theta \text{ }y.A_j
        \end{aligned}\end{equation}
    Il predicato è costruito con i soliti operatori AND, OR, NOT, IMPLICAZIONE e in più, dai quantificatori $\forall$ e $\exists$.}
    \end{itemize}
\textbf{ES1:} elencare i pazienti residenti a Torino
    \begin{equation}\begin{aligned}
        \textbf{T}: p.Nome, p.Cognome\\
        \textbf{L}: p(PAZIENTI)\\
        \textbf{F}: p.Residenza = 'TO'
    \end{aligned}\end{equation}
\textbf{ES2:} elencare i dati dei pazienti ricoverati nel reparto $A$ (si vuole anche la data del ricovero)
    \begin{equation}\begin{aligned}
        \textbf{T}: p.Nome, p.Cognome, r.Inizio\\
        \textbf{L}: p(PAZIENTI), r(RICOVERI)\\
        \textbf{F}: p.COD = r.PAZ \wedge Reparto = 'A'
    \end{aligned}\end{equation}
    
\subsection{Join nel calcolo relazionale}
Nel calcolo relazionale, il join si costruisce semplicemente inserendo nella formula la condizione di confronto delle variabili (relazioni) coinvolte, preliminarmente dichiarate nella Range List.

\subsection{Quantificazione universale ed esistenziale}
Quando una variabile non è espressa nella Target List, possiamo riformulare l'interrogazione usando la quantificazione esistenziale o universale.\\
La sintassi utilizzata è la seguente:
    \begin{equation}\begin{aligned}
        \exists \text{ } variabile(Relazione)(formula)\\
        \forall \text{ } variabile(Relazione)(formula)\\
    \end{aligned}\end{equation}
La formula è un predicato del primo ordine che può contenere sia variabili libere che quantificate, tuttavia le uniche variabili libere presenti nella formula devono essere dichiarate nella Range List.\\\\
\textbf{ES1:} elencare i dati dei pazienti ricoverati nel reparto $A$ (non è richiesta la data di inizio ricovero):
    \begin{equation}\begin{aligned}
        \textbf{T}: p.Nome, p.Cognome\\
        \textbf{L}: p(PAZIENTI)\\
        \textbf{F}: \exists r(RICOVERI)(p.COD = r.PAZ \wedge Reparto = 'A')
    \end{aligned}\end{equation}
\textbf{ES2:} elencare i medici curanti del paziente $A102$.
    \begin{equation}\begin{aligned}
        \textbf{T}: m.Nome, m.Cognome\\
        \textbf{L}: m(MEDICI)\\
        \textbf{F}: \exists r(RICOVERI)(r.Reparto = m.Reparto \wedge r.PAZ = 'A102')
    \end{aligned}\end{equation}
\textbf{ES3:} elencare i medici curanti del paziente Rossini Piero.
    \begin{equation}\begin{aligned}
        \textbf{T}: m.Nome, m.Cognome\\
        \textbf{L}: m(MEDICI)\\
        \textbf{F}: \exists p(PAZIENTI)\\
        (\exists r(RICOVERI)(p.COD = r.PAZ 
        \wedge  r.Reparto = m.Reparto)\\
        \wedge p.Nome = 'Piero' \wedge p.Cognome = 'Rossini' )
    \end{aligned}\end{equation}
\textbf{ES4:} elencare cognome e nome comune a pazienti e medici.
    \begin{equation}\begin{aligned}
        \textbf{T}: p.Nome, p.Cognome\\
        \textbf{L}: p(PAZIENTI)\\
        \textbf{F}: \exists m(MEDICI)(m.Nome = p.Nome \wedge m.Cognome = p.Cognome)
    \end{aligned}\end{equation}
\textbf{ES5:} elencare i cognomi e i nomi dei pazienti non medici.\\
È sufficiente anteporre il NOT alla quantificazione esistenziale dell'esercizio precedente.
    \begin{equation}\begin{aligned}
        \textbf{T}: p.Nome, p.Cognome\\
        \textbf{L}: p(PAZIENTI)\\
        \textbf{F}: \neg\exists m(MEDICI)(m.Nome = p.Nome \wedge m.Cognome = p.Cognome)
    \end{aligned}\end{equation}
\textbf{ES6:} elencare i pazienti con almeno un ricovero in ogni reparto.
    \begin{equation}\begin{aligned}
        \textbf{T}: p.Nome, p.Cognome\\
        \textbf{L}: p(PAZIENTI)\\
        \textbf{F}: \forall r(REPARTI) \exists r'(RICOVERI)(p.COD = r'.PAZ \wedge r.COD = r'.Reparto)
    \end{aligned}\end{equation}
\textbf{ES7:} elencare i pazienti ricoverati due o più volte.
    \begin{equation}\begin{aligned}
        \textbf{T}: p.Nome, p.Cognome\\
        \textbf{L}: p(PAZIENTI)\\
        \textbf{F}: \exists r'(RICOVERI)(\exists r''(RICOVERI) (p.COD = r'.PAZ \wedge p.COD = r''.PAZ\\
        \wedge r'.Inizio \neq r''.Inizio))
    \end{aligned}\end{equation}
    
\subsection{Negazione essenziale}
Lo schema generale di soluzione di interrogazioni contenenti una negazione essenziale è
    \begin{equation}\begin{aligned}
        R = U - P
    \end{aligned}\end{equation}
In calcolo relazionale, esprimiamo questo schema con il seguente pattern di soluzione:
    \begin{equation}\begin{aligned}
        \textbf{F} = formulaU \wedge \not formulaP 
    \end{aligned}\end{equation}
\textbf{ES1:} elencare i pazienti ricoverati una sola volta.
    \begin{equation}\begin{aligned}
        \textbf{T}: p.Nome, p.Cognome\\
        \textbf{L}: p(PAZIENTI)\\
        \textbf{F}: 
        \exists r(RICOVERI) (p.COD = r.PAZ) \wedge \\
        \neg\exists r'(RICOVERI)(\exists r''(RICOVERI) (p.COD = r'.PAZ \wedge p.COD = r''.PAZ)
    \end{aligned}\end{equation}
\textbf{ES2:} elencare i medici non primari.
    \begin{equation}\begin{aligned}
        \textbf{T}: m.Nome, m.Cognome\\
        \textbf{L}: m(MEDICI)\\
        \textbf{F}: \neg \exists r(REPARTI)
        (r.Primario = m.MATR)
    \end{aligned}\end{equation}
In questo caso apparentemente manca l'universo del discorso, ma è implicitamente definito nella variabile libera $m$.

\subsection{Prodotto cartesiano}
Il prodotto cartesiano viene reso utilizzando la seguente sintassi:
    \begin{equation}\begin{aligned}
        \textbf{T}: x.*, y*\\
        \textbf{L}: x(R), y(S)\\
    \end{aligned}\end{equation}

\subsection{Unione}
L'unione insiemistica dell'algebra relazionale non è esprimibile con il calcolo relazionale con dichiarazione di range.

\subsection{Limiti del calcolo relazionale}
Il problema dell'inesprimibilità dell'unione insiemistica non riguarda tutti i linguaggi dichiarativi, ma soltanto il calcolo relazionle su tuple con dichiarazione di range.\\
Per superare questa limitazione, viene introdotto il \textbf{calcolo relazionale sui domini}.

\subsubsection{Esempio}
Consideriamo la seguente base dati:
    \begin{equation}\begin{aligned}
        S (\text{Studenti}) = (\underline{MATR}, Nome, Indirizzo)\\
        E (\text{Esami}) = (\underline{MATR, Corso, Indirizzo})\\
        O (\text{Offerta Formativa}) = (\underline{Corso, Indirizzo})\\
    \end{aligned}\end{equation}
Elencare gli studenti che hanno superato tutti gli esami del loro indirizzo.
    \begin{equation}\begin{aligned}
        \textbf{T}: s.*\\
        \textbf{L}: s(S)\\
        \textbf{F}: \forall o(O) (o.Indirizzo = s.Indirizzo \\
        \rightarrow \exists e(E) (e.MATR = s.MATR \wedge e.Corso = o.Corso \wedge e.Indirizzo = o.Indirizzo))
    \end{aligned}\end{equation}
    \begin{itemize}
        \item{Con la quantificazione universale scorro tutti i corsi dell'offerta formativa.}
        \item{Con l'antecedente mi soffermo sui corsi dell'indirizzo dello studente.}
        \item{Con il conseguente verifico che lo studente abbia superato l'esame}
        \item{Se scorro corsi che non sono dell'indirizzo dello studente, l'antecedente è falso e l'implicazione rimane vera.}
    \end{itemize}

\subsubsection{Esempio}
Consideriamo la seguente base dati:
    \begin{equation}\begin{aligned}
        A (\text{Autori}): (\underline{COD}, Cognome, Nome)\\
        P (\text{Pubblicazioni}): (\underline{ID}, Titolo, Anno)\\
        F (\text{Firma}): (\underline{A, P})
    \end{aligned}\end{equation}
Trovare gli autori che hanno pubblicato solo in collaborazione (autori che non hanno mai pubblicato una pubblicazione di cui sono gli unici firmatari)
    \begin{equation}\begin{aligned}
        \textbf{T}: a.*\\
        \textbf{L}: a(A)\\
        \textbf{F}: \forall f(F) (f.A = a.COD \rightarrow \exists f'(F) (f'.P = f.P \wedge f'.A \neq a.COD))
    \end{aligned}\end{equation}