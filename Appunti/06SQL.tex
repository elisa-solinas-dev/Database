\chapter{Introduzione}

Il linguaggio SQL è il linguaggio di riferimento per tutti i DBMS relazionali, e implementa sia il DDL (Data Definition Language) che il DML (Data Manipulation Language).\\\\
SQL è un linguaggio che ci permette di interrogare e gestire le basi di dati.\\
Ciascuna interrogazione è implementata per mezzo di costrutti di programmazione detti \textbf{query}.\\\\
Per il linguaggio SQL sono stati realizzati diversi standard e diverse implementazioni.

\section{PostgreSQL}
PostgreSQL è un DBMS opensource, conforme agli standard SQL, che aggiunge ai database relazionali caratteristiche che lo rendono classificabile come object-relational.\\
Aggiunge inoltre funzionalità quali vincoli, trigger e rules.\\\\
PostgreSQL è basato su un'architettura Client/Server, ovvero la connessione è frutto di due processi:
    \begin{itemize}
        \item{Il processo \textbf{server}, chiamato postgres o postmaster, che accetta le connessioni dai client e ha il compito di interagire con i database su delega del client che ha fatto richiesta. Spesso gira su un altro computer, più potente, e accetta richieste contemporaneamente da più client.}
        \item{Un \textbf{client} che richiede di effettuare determinate operazioni su uno o più database. Un client può essere: un’applicazione grafica, una pagina web, un programma scritto da un utente. pgAdmin, l'utility che permette di impartire comandi SQL interattivamente, è un esempio di client.}
    \end{itemize}

\section{Domini dei dati}
\subsection{Caratteri e stringhe}
Le \textbf{stringhe a lunghezza fissa} vengono dichiarate con la seguente sintassi:
    \begin{lstlisting}{Stringhe}
        char(length)
    \end{lstlisting}
Le \textbf{stringhe a lunghezza variabile} vengono dichiarate con la seguente sintassi:
    \begin{lstlisting}{Stringhe}
        varchar(length)
    \end{lstlisting}
Quando la lunghezza di una stringa non viene specificata, si assume che sia 1 (\textbf{singolo carattere}).\\\\
\textbf{Esempi}:
    \begin{itemize}
        \item{$char(100)$: stringa di esattamente 100 caratteri}
        \item{$varchar(100)$: stringa di lunghezza variabile fino a un massimo di 100 caratteri}
        \item{$char$: un singolo carattere}
    \end{itemize}
    
\subsection{Tipi numerici esatti}
I numeri decimali vengono rappresentati in maniera esatta con una delle seguenti sintassi:
    \begin{lstlisting}{Numeri decimali}
        decimal (precisione, scala)
        numeric (precisione, scala)
   \end{lstlisting}
    \begin{itemize}
        \item{$precisione$: numero totale di cifre decimali (opzionale). Per $decimal$ la precisione specificata è quella minima garantita, per $numeric$ è quella esatta.}
        \item{$scala$: numero di cifre decimali dopo la virgola (opzionale).}
    \end{itemize}
I numeri interi vengono rappresentati con una delle seguenti sintassi:
    \begin{lstlisting}{Numeri interi}
        integer
        smallint
    \end{lstlisting}
La rappresentazione dei numeri interi è dipendente dalla specifica implementazione.\\\\
\textbf{Esempi}:
    \begin{itemize}
        \item{$decimal(6,4)$: decimali da -99.9999 a +99.999}
        \item{$numeric(4)$: decimali da -9999 a +9999}
        \item{$integer$: interi a 32 bit (solitamente)}
    \end{itemize}

\subsection{Tipi numerici approssimati}
I numeri in virgola mobile con mantissa ed esponente vengono rappresentati con la seguente sintassi:
    \begin{lstlisting}{Numeri approssimati}
        float (precisione)
        real
        double precision
    \end{lstlisting}
    \begin{itemize}
        \item{$precisione$: numero di cifre binarie per la mantissa}
    \end{itemize}

\subsection{Date}
Possiamo usare la seguente sintassi per rappresentare istanti temporali:
    \begin{itemize}
        \item{$date$: per le date. Comprende i sottocampi $year$, $month$, $day$.}
        \item{$time(precisione)$: per gli orari. Comprende i sottocampi $hour$, $minute$, $second$.}
        \item{$timestamp(precisione)$: per specificare sia date che orari. Comprende i sottocampi $year$, $month$, $day$, $hour$, $minute$, $second$.}
    \end{itemize}
    \begin{itemize}
        \item{$precisione$: numero di cifre per le frazioni di secondo (opzionale)}
    \end{itemize}
\textbf{Esempi}:
    \begin{itemize}
        \item{$timestamp(2)$: per memorizzare eventi di log con precisione al centesimo di secondo.}
    \end{itemize}
    
\subsection{Intervalli}
Per rappresentare gli intervalli temporali, usiamo la sintassi:
    \begin{lstlisting}{Intervalli Temporali}
        interval unita1 [to unita2]
    \end{lstlisting}
    \begin{itemize}
        \item{$unita1$: unità di tempo più grossolana}
        \item{$unita2$ (opzionale): unità di tempo più fine}
    \end{itemize}
\textbf{Esempi}:
    \begin{itemize}
        \item{$interval$ $year$: esprime intervalli in anni.}
        \item{$interval$ $year$ $to$ $month$: esprime intervalli misurati in anni e mesi (e.g. un anno e due mesi).}
        \item{$interval$ $day$ $to$ $second(2)$: esprime intervalli misurati in giorni, ore, minuti, secondi e centesimi di secondo.}
        \item{$interval$ $year$ $to$ $minute$: non è permesso (non si può passare in modo preciso da $month$ a $day$ perché i mesi hanno durata diversa.}
    \end{itemize}

\subsection{Altri domini}
    \begin{itemize}
        \item{$boolean$: valori booleani.}
        \item{$bigint$: interi "grossi"}
        \item{$blob$: Binary Large OBject (immagini, video, file di vario tipo) (in PostgreSQL $bytea$ oppure $lo$).}
        \item{$clob$: Character Large OBject (lunghi file di testo) (in PostgreSQL $text$).}
    \end{itemize}

\subsection{Domini personalizzati}
È possibile creare domini personalizzati (le parti tra $[]$ sono opzionali:
    \begin{lstlisting}{Domini personalizzati}
        create domain NomeDominio as TipoDato
        [default ValoreDefault] [Vincoli]
    \end{lstlisting}
    \begin{itemize}
        \item{$NomeDominio$: nome del nuovo dominio}
        \item{$TipoDato$: nome del tipo di base}
        \item{$default \quad ValoreDefault$: valore asseegnato in automatico, se non specificato esplicitamente (opzionale)}
        \item{$Vincoli$: insieme di vincoli sui valori assunti dal dominio personalizzato (opzionale)}
    \end{itemize}
I domini personalizzati sono comunque semplici (no array, no struct, no record).\\\\
\textbf{Esempio 1}: voto di un esame
    \begin{lstlisting}{Domini personalizzati - Esempio 1}
        create domain Voto
        as smallint default NULL
        check (value >= 18) AND (value <= 30);
    \end{lstlisting}
\textbf{Esempio 2}: temperatura corporea (gradi Celsius)
    \begin{lstlisting}{Domini personalizzati - Esempio 2}
        create domain Temperatura
        as decimal(3, 1) default NULL
        check (value >= 35) AND (value <= 42);
    \end{lstlisting}
L'utilità dei nuovi domini è legata alla possibilità di astrazione.\\\\
\textbf{Esempio 3}: per passare dai gradi Celsius ai gradi Fahrenheit è sufficiente cambiare una volta per tutte il dominio di Temperatura:
    \begin{lstlisting}{ Domini personalizzati - Esempio 3}
        create domain Temperatura
        as decimal(4, 1) default NULL
        check (value >= 95) AND (value <= 107.6);
    \end{lstlisting}

\section{Definizione di tabelle}
Per creare e definire una nuova tabella utilizziamo la seguente sintassi:
    \begin{lstlisting}{Creare nuova tabella}
        create table NomeTabella(
            NomeAttributo1 Dominio1 [ValoreDefault1][Vincoli1]
            ...
            Nome AttributoN DominioN [ValoreDefaultN][VincoliN]
        );
    \end{lstlisting}
\textbf{Esempio}: $Dipartimento(\underline{Nome}, Indirizzo, Citta)$
    \begin{lstlisting}{Esempio Dipartimento}
        create table Dipartimento(
            Nome varchar(20) primary key,
            Indirizzo varchar(50),
            Citta varchar(20) not null
        );
    \end{lstlisting}
    
\subsection{Valori di default}
Per ogni attributo può essere specificato un valore predefinito che verrà usato se, nell'inserimento di una riga, non viene specificato un valore per quell'attributo.\\\\
\textbf{Esempio}: 
    \begin{lstlisting}{Esempio Valori Default}
        NumeroFigli smallint default 0,
        Email varchar(64) default 'unito.it',
        StatoCivile varchar(20) default 'libero'
    \end{lstlisting}
Se non si specifica esplicitamente un valore di default, viene usato $null$.

\section{Definizione dei vincoli}
I vincoli servono a definire proprietà che devono essere verificate da ogni istanza della base di dati per garantirne l'integrità (vincoli di integrità).\\
Si differenziano in:
    \begin{itemize}
        \item{\textbf{Vincoli intrarelazionali}: relativi a una sola tabella}
        \item{\textbf{Vincoli interrelazionali}: relativi a più tabelle}
    \end{itemize}
Si possono specificare:
    \begin{itemize}
        \item{Contestualmente alla definizione degli attributi}
        \item{Alla fine della definizione della tabella}
    \end{itemize}
Si possono usare vincoli predefiniti oppure si può specificare l'espressione logica che il vincolo deve verificare.\\\\
In alcune occasioni può essere utile assegnare un nome a un vincolo. Per farlo si utilizza la seguente sintassi dopo tutte le definizioni di attributi:
    \begin{lstlisting}
        constraint NomeVincolo DefinizioneVincolo
    \end{lstlisting}

\subsection{Vincoli intrarelazionali predefiniti}
\textbf{Vincolo $not$ $null$:}
    \begin{itemize}
        \item{È un vincolo di tupla che indica che il valore nullo non è ammesso come valore per un determinato attributo, quindi rende l'attributo obbligatorio}
        \item{Se l'attributo non viene specificato in fase di inserimento e non si è specificato un valore di default, si viola il vincolo d'integrità e l'operazione è annullata}
        \item{Sintassi
            \begin{lstlisting}
                NomeAttributo Dominio {}not null
            \end{lstlisting}}
        \item{Se per l'attributo viene specificato un valore di default è possibile effettuare l'inserimento anche senza specificarne il valore.}
    \end{itemize}
\textbf{Esempio}
    \begin{lstlisting}
        Citta varchar(20) not null
    \end{lstlisting}
    
\textbf{Vincolo $unique$:}
    \begin{itemize}
        \item{È un vincolo di tabella che indica che il valore di un attributo o le combinazioni di valori su un insieme di attributi sono una superchiave:
            $$righe \text{ } diverse \leftrightarrow 
            valori \text{ } diversi$$}
        \item{Fa eccezione il valore nullo (che può comparire in più righe senza violare il vincolo).}
        \item{Sintassi (vincolo su un solo attributo):
            \begin{lstlisting}
                NomeAttributo Dominio unique
            \end{lstlisting}}
    \end{itemize}
\textbf{Esempio}
    \begin{lstlisting}
        Matricola varchar(6) unique
    \end{lstlisting}
    
\textbf{Vincolo $unique$ su insiemi di attributi}:
    \begin{itemize}
        \item{Va specificato dopo la definizione degli attributi della tabella.}
        \item{Sintassi:
            \begin{lstlisting}
                NomeAttributo1 Dominio1
                NomeAttributo2 Dominio2
                ...
                unique (NomeAttributo1, NomeAttributo2)
            \end{lstlisting}}
    \end{itemize}
\textbf{Esempio}
    \begin{lstlisting}
        Nome varchar(255),
        Cognome varchar(255),
        unique(Nome, Cognome)
    \end{lstlisting}
    
\textbf{Vincolo $primary$ $key$ di chiave primaria}:
    \begin{itemize}
        \item{È un vincolo di tabella che indica che un attributo o un insieme di attributi sono la chiave primaria.}
        \item{Gli attributi così definiti non possono assumere valore nullo.}
        \item{Può esserci un solo vincolo primary key per ogni tabella.}
        \item{Vincolo $primary$ $key$ su un solo attributo: va specificato nella definizione dell'attributo mediante la seguente sintassi:
            \begin{lstlisting}
                NomeAttributo Dominio primary key
            \end{lstlisting}}
        \item{Vincolo $primary$ $key$ su un insieme di attributi: va specificato dopo la definizione degli attributi mediante la seguente sintassi:
            \begin{lstlisting}
                NomeAttributo1 Dominio1,
                NomeAttributo2 Dominio2,
                ...
                primary key (NomeAttributo1, NomeAttributo2, ...)
            \end{lstlisting}}
    \end{itemize}

\subsubsection{Vincoli interrelazionali predefiniti}
Un \textbf{vincolo di integrità referenziale} (di \textbf{chiave esterna}):
    \begin{itemize}
        \item{Crea un vincolo tra i valori uno o più attributi della tabella (interna) su cui è definito e uno o più attributi di un'altra tabella (esterna).}
        \item{Per ogni riga della tabella interna, il valore dell'attributo specificato nel vincolo, se diverso da $null$, dev'essere presente tra i valori del corrispondente attributo della tabella esterna.}
        \item{Sintassi con un solo attributo:
            \begin{lstlisting}
                NomeAttributo references NomeTabellaEsterna(NomeAttributoEsterno)
            \end{lstlisting}}
        \item{Sintassi con più attributo:
            \begin{lstlisting}
                NomeAttributo1 Dominio1,
                NomeAttributo2 Dominio2,
                ...
                foreign key (NomeAttributo1, NomeAttributo2, ...)
                    references NomeTabellaEsterna (NomeAttributoEsterno1, Nome AttributoEsterno2, ...)
            \end{lstlisting}
            L'ordine è importante! Ciascun attributo viene mappato al corrisponente secondo l'ordine con cui lo specifichiamo.}
    \end{itemize}

\subsubsection{Violazione del vincolo di integrità referenziale}
In generale, quando un vincolo viene violato, il DBMS rifiuta l'operazione che causerebbe la violazione e segnala un errore.\\
Con i vincoli di integrità referenziale si possono specificare altre reazioni da adottare in caso di violazione.\\\\
Casi in cui può avvenire una violazione di un vincolo di integrità referenziale:
    \begin{enumerate}
        \item{Inserimento di una nuova riga nella tabella interna.}
        \item{Modifica, nella tabella interna, di un valore dell'attributo referente.}
        \item{Modifica, nella tabella esterna, di un valore dell'attributo riferito.}
        \item{Cancellazione di una riga nella tabella esterna.}
    \end{enumerate}
I primi due casi riguardano un cambiamento della \textbf{tabella interna}, gli ultimi due un cambiamento della \textbf{tabella esterna}.\\\\
Quando cambia la tabella interna (casi 1 e 2), l'operazione viene semplicemente \textbf{rifiutata}.\\
Quando cambia la tabella esterna (casi 3 e 4), si può specificare quali variazioni apportare alla tabella interna.\\
Questa asimmetria nel comportamento deriva dal fatto che dal punto di vista applicativo la tabella esterna ricopre il ruolo di tabella principale (\textbf{master}) e quella interna di secondaria (\textbf{slave}) che deve adeguarsi alle variazioni.\\\\
In seguito a una \textbf{modifica} (caso 3), le reazioni possono essere le seguenti:
    \begin{enumerate}
        \item{\textbf{cascade}: il nuovo valore dell'attributo della tabella esterna viene riportato in tutte le corrispondenti righe della tabella interna.}
        \item{\textbf{set null}: all'attributo referente viene assegnato il valore nullo al posto del valore modificato nella tabella esterna.}
        \item{\textbf{set default}: all'attributo referente viene assegnato il valore di default al posto del valore modificato nella tabella esterna.}
        \item{\textbf{no action}: nessuna reazione (e quindi la modifica non viene consentita).}
    \end{enumerate}
In seguito a una \textbf{cancellazione} (caso 4), le reazioni possono essere le seguenti:
    \begin{enumerate}
        \item{\textbf{cascade}: tutte le righe della tabella interna corrispondenti alla riga cancellata nella tabella esterna vengono cancellate.}
        \item{\textbf{set null}: all'attributo referente viene assegnato il valore nullo al posto del valore cancellato nella tabella esterna.}
        \item{\textbf{set default}: all'attributo referente viene assegnato il valore di default al posto del valore cancellato nella tabella esterna.}
        \item{\textbf{no action}: nessuna reazione (e quindi la cancellazione non viene consentita).}
    \end{enumerate}
La sintassi per specificare il comportamento in caso di violazione di un vincolo interrelazionale è la seguente:
    \begin{lstlisting}
        //Per la modifica, subito dopo il vincolo
        on update Reazione
        //Per la cancellazione, subito dopo il vincolo
        on delete Reazione
    \end{lstlisting}

\subsubsection{Esempio}
    \begin{lstlisting}{Creazione tabella Ricoveri}
        create table Ricoveri (
            PAZ varchar(4),
            Inizio date,
            Fine date,
            Reparto char,
            primary key (PAZ, Inizio),
            foreign key (PAZ) references Pazienti(COD)
                on update cascade
                on delete no action,
            foreign key (Reparto) references Reparti (COD)
                on update cascade
                on delete set null
        );
    \end{lstlisting}

\section{Modifiche alle definizioni}
SQL permette di modificare la definizione di tabelle, domini, vincoli precedentemente introdotti:
    \begin{lstlisting}{Modifica definizioni}
        alter (modifica),
        drop (cancellazione),
        add (aggiunta)
    \end{lstlisting}
