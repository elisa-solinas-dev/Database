\chapter{Operatori Insiemistici}

SQL mette a disposizione anche degli operatori insiemistici, simili a quelli disponibili nell'algebra relazionale.\\
Gli operatori disponibili sono quelli di  \texttt{union, intersect, except}, aventi significato analogo ai corrispettivi in algebra relazionale.\\\\
Notiamo che ogni interrogazione che faccia uso di \texttt{intersect} ed \texttt{except} può essere espressa utilizzando altri costrutti del linguaggio (tipicamente usando interrogazioni nidificate, esposte nel capitolo 12). Al contrario, il comando \texttt{union} arricchisce il potere espressivo di SQL e permette di scrivere interrogazioni altrimenti non formulabili.\\\\
La sintassi per l'uso degli operatori insiemistici è la seguente:
	\begin{lstlisting}
		select ...SQL...
		<union | intersect | except> [all]
		select ...SQL...	
	\end{lstlisting}
Gli operatori insiemistici, al contrario degli altri costrutti, assumono come default di eseguire un'eliminazione dei duplicati, interpretando meglio il tipico significato matematico di questi operatori. Qualora nell'interrogazione si voglia adottare una diversa interpretazione e si vogliano preservare i duplicati, è sufficiente specificare il parametro \texttt{all}.\\\\
SQL, a differenza dell'algebra relazionale, non richiede che gli schemi su cui vengono effettuate le operazioni insiemistiche siano identici, ma solo che gli attributi siano in pari numero e abbiano domini compatibili. La corrispondenza tra gli attributi non si basa sul nome ma sulla loro posizione: se gli attributi hanno nome diverso, il risultato normalmente usa i nomi del primo operando.

\subsubsection{Esempio 1}
Consideriamo la seguente base dati:
	\begin{equation}\begin{aligned}
		IMPIEGATO (\underline{Nome, Cognome}, Dipartimento, Ufficio, Stipendio, Citta)\\
		DIPARTIMENTO (\underline{Nome}, Indirizzo, Citta)
	\end{aligned}\end{equation}
Estrarre nomi e cognomi degli impiegati.
	\begin{lstlisting}{Esempio 1}
		select
			I.Nome
		from
			Impiegato as I
		
		union
		
		select
			I.Cognome
		from
			Impiegato as I
	\end{lstlisting}
	
\subsubsection{Esempio 2}
Consideriamo la seguente base dati:
	\begin{equation}\begin{aligned}
		IMPIEGATO (\underline{Nome, Cognome}, Dipartimento, Ufficio, Stipendio, Citta)\\
		DIPARTIMENTO (\underline{Nome}, Indirizzo, Citta)
	\end{aligned}\end{equation}
Estrarre i nomi e i cognomi di tutti gli impiegati, eccetto quelli appartenenti al dipartimento $Amministrazione$, mantenendo i duplicati.
	\begin{lstlisting}{Esempio 2}
		select 
			I.Nome
		from
			Impiegato as I
		where 
			I.Dipartimento <> 'Amministrazione'
			
			union all
			
		select
			I.Cognome
		from
			Impiegato as I
		where
			I.Dipartimento <> 'Amministrazione
	\end{lstlisting}

\subsubsection{Esempio 3}
Consideriamo la seguente base dati:
	\begin{equation}\begin{aligned}
	IMPIEGATO (\underline{Nome, Cognome}, Dipartimento, Ufficio, Stipendio, Citta)\\
	DIPARTIMENTO (\underline{Nome}, Indirizzo, Citta)
	\end{aligned}\end{equation}
Estrarre i cognomi di impiegati che sono anche nomi
	\begin{lstlisting}{Esempio 3}
		select 
		I.Nome
		from
		Impiegato as I
	
		intersect
	
		select
		I.Cognome
		from
		Impiegato as I
	\end{lstlisting}

\subsubsection{Esempio 3}
Consideriamo la seguente base dati:
	\begin{equation}\begin{aligned}
		IMPIEGATO (\underline{Nome, Cognome}, Dipartimento, Ufficio, Stipendio, Citta)\\
		DIPARTIMENTO (\underline{Nome}, Indirizzo, Citta)
	\end{aligned}\end{equation}
Estrarre i nomi degli impiegati che non sono cognomi di qualche impiegato
	\begin{lstlisting}{Esempio 3}
		select 
		I.Nome
		from
		Impiegato as I
		
		except
		
		select
		I.Cognome
		from
		Impiegato as I
	\end{lstlisting}

