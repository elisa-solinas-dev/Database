\chapter{Aggregati}




\section{Operatori aggregati}
Gli operatori aggregati costituiscono una delle più importanti estensioni di SQL rispetto all'algebra relazionale. Infatti, in algebra relazionale tutte le condizioni vengono valutate su una tupla alla volta, la condizione è sempre un predicato che viene valutato su ciascuna tupla indipendentemente da tutte le altre.\\
Spesso, però, nei contesti reali è richiesto di valutare delle proprietà che dipendono da insiemi di tuple.\\\\
Gli operatori aggregati vengono gestiti come un'estensione delle normali interrogazioni. Prima viene normalmente eseguita l'interrogazione, considerando solo le parti \texttt{from} e \texttt{where}.\\
L'operatore aggregato viene poi applicato alla tabella contenente il risultato dell'interrogazione.


\subsection{Operatore \texttt{count}}
L'operatore \texttt{count} calcola il numero di tuple che compongono la tabella risultante dall'interrogazione, usando la seguente sintassi:
    $$\texttt{count (<* | [distinct|all] listaAttributi>)}$$
La prima opzione \texttt{*} restituisce il numero di righe.\\
L'opzione \texttt{distinct} restituisce il numero di diversi valori degli attributi in \texttt{ListaAttributi}.\\
L'opzione \texttt{all} restituisce il numero di righe che possiedono valori diversi dal valore nullo per gli attributi in \texttt{ListaAttributi}.\\
Se si specifica un attributo e si omette \texttt{distinct} o \texttt{all}, si assume \texttt{all} come default.
    
\subsubsection{Esempio 1}
Consideriamo la seguente base dati:
    \begin{equation}\begin{aligned}
        IMPIEGATO (\underline{Nome, Cognome}, Dipartimento, Ufficio, Stipendio, Citta)\\
        DIPARTIMENTO (\underline{Nome}, Indirizzo, Citta)
    \end{aligned}\end{equation}
Estrarre il numero di impiegati del dipartimento $Produzione$
    \begin{lstlisting}{Esempio 1}
        select 
            count(*)
        from 
            Impiegato as I
        where 
            I.Dipartimento = 'Produzione'
    \end{lstlisting}

\subsubsection{Esempio 2}
Consideriamo la seguente base dati:
    \begin{equation}\begin{aligned}
        IMPIEGATO (\underline{Nome, Cognome}, Dipartimento, Ufficio, Stipendio, Citta)\\
        DIPARTIMENTO (\underline{Nome}, Indirizzo, Citta)
    \end{aligned}\end{equation}
Estrarre il numero di diversi valori dell'attributo $Stipendio$ fra tutte le righe di $IMPIEGATO$
    \begin{lstlisting}{Esempio 2}
        select 
            count (distinct I.Stipendio)
        from 
            Impiegato as I
    \end{lstlisting}

\subsubsection{Esempio 3}
Consideriamo la seguente base dati:
    \begin{equation}\begin{aligned}
        IMPIEGATO (\underline{Nome, Cognome}, Dipartimento, Ufficio, Stipendio, Citta)\\
        DIPARTIMENTO (\underline{Nome}, Indirizzo, Citta)
    \end{aligned}\end{equation}
Estrarre il numero di righe che possiedono un valore non nullo per l'attributo $Nome$
    \begin{lstlisting}{Esempio 3}
        select 
            count (all I.Nome)
        from 
            Impiegato as I
    \end{lstlisting}

\subsection{Altri operatori aggregati}
Gli altri operatori aggregati (\texttt{sum, max, min, avg}), invece, ammettono come argomento un attributo o un'espressione, eventualmente preceduta dalle parole chiave \texttt{distinct} o \texttt{all}.\\
Le funzioni aggregate \texttt{sum} e \texttt{avg} ammettono come argomento solo espressioni che rappresentano valori numerici o intervalli di tempo.\\
Le funzioni \texttt{min} e \texttt{max} richiedono solamente che sull'espressione sia definito un ordinamento, per cui si possono applicare anche su stringhe di caratteri o su istanti di tempo
	$$\texttt{<sum | max | min | avg> ([distinct | all] AttEsp)}$$
Gli operatori si applicano sulle righe che soddisfano la condizione presente nella clausola \texttt{where} e hanno il seguente significato:
	\begin{itemize}
		\item{\texttt{sum}: restituisce la somma dei valori posseduti dall'espressione}
		\item {\texttt{max} e \texttt{min}: restituiscono rispettivamente il valore massimo e minimo}
		\item {\texttt{avg}: restituisce la medai dei valori (ossia il risultato della divisione di \texttt{sum} per \texttt{count})}
	\end{itemize}
Le parole chiave \texttt{distinct} e \texttt{all} hanno il significato che abbiamo già visto: \texttt{distinct} elimina i duplicati, mentre \texttt{all} trascura solo i valori nulli; l'uso di \texttt{distinct} o \texttt{all} con gli operatori \texttt{max} e \texttt{min} non ha effetto sul risultato.

\subsubsection{Esempio 4}
Consideriamo la seguente base dati:
	\begin{equation}\begin{aligned}
		IMPIEGATO (\underline{Nome, Cognome}, Dipartimento, Ufficio, Stipendio, Citta)\\
		DIPARTIMENTO (\underline{Nome}, Indirizzo, Citta)
	\end{aligned}\end{equation}
Estrarre la somma degli stipendi del dipartimento 'Amministrazione'
	\begin{lstlisting}{Esempio 4}
		select 
			sum(I.Stipendio)
		from 
			Impiegato as I
		where
			I.Dipartimento = 'Amministrazione'
	\end{lstlisting}
	


\section{Interrogazioni con raggruppamento}
Abbiamo caratterizzato gli operatori aggregati come gli operatori che vengono applicati a un insieme di righe. Molto spesso, però, sorge l'esigenza di applicare l'operatore aggregato separatamente a sottoinsiemi di righe.\\
Per poter utilizzare in questo modo l'operatore aggregato, SQL mette a disposizione la clausola \texttt{group by}, che permette di specificare come dividere le tabelle in sottoinsiemi. La clausola ammette come argomento un insieme di attributi e l'interrogazione raggrupperà le righe che possiedono gli stessi valori per questo insieme di attributi.\\\\
Inizialmente, l'interrogazione viene eseguita come se la clausola non esistesse, selezionando gli attributi che compaiono come argomenti della clausola \texttt{group by} o che compaiono all'interno dell'espressione argomento dell'operatore aggregato.\\
La tabella ottenuta viene poi analizzata, dividendo le righe in insiemi caratterizzati dallo stesso valore degli attributi che compaiono come argomento della clausola \texttt{group by}.\\
Dopo che le righe sono state raggruppate in sottoinsiemi, l'operatore aggregato viene applicato separatamente su ogni sottoinsieme. Il risultato dell'interrogazione è costituito da una tabella con righe che contengono l'esito della valutazione dell'operatore aggregato affiancato al valore dell'attributo che è stato usato per l'aggregazione.\\\\
La sintassi impone che, in un'interrogazione che fa uso della clausola \texttt{group by}, possa comparire come argomento della \texttt{select} solamente un sottoinsieme degli attributi usati nella clausola \texttt{group by}. Per questi attributi, infatti, ciascuna tupla del sottoinsieme sarà caratterizzata dallo stesso valore.

\subsubsection{Esempio 5}
Consideriamo la seguente base dati:
	\begin{equation}\begin{aligned}
		IMPIEGATO (\underline{Nome, Cognome}, Dipartimento, Ufficio, Stipendio, Citta)\\
		DIPARTIMENTO (\underline{Nome}, Indirizzo, Citta)
	\end{aligned}\end{equation}
Estrarre la somma degli stipendi di tutti gli impiegati dello stesso dipartimento.
	\begin{lstlisting}{Esempio 5}
		select 
			I.Dipartimento, sum(I.Stipendio)
		from 
			Impiegato as I
		group by
			I.Dipartimento
	\end{lstlisting}

\subsubsection{Esempio 6}
Consideriamo la seguente base dati:
	\begin{equation}\begin{aligned}
		IMPIEGATO (\underline{Nome, Cognome}, Dipartimento, Ufficio, Stipendio, Citta)\\
		DIPARTIMENTO (\underline{Nome}, Indirizzo, Citta)
	\end{aligned}\end{equation}
Estrarre i dipartimenti, il numero di impiegati di ciascun dipartimento, e la città in cui il dipartimento ha sede.
	\begin{lstlisting}{Esempio 6}
		select 
			I.Dipartimento
			count(*)
			D.Citta
		from 
			Impiegato as I
			join Dipartimento D
				on (I.Dipartimento = D.Nome)
		group by
			I.Dipartimento, D.Citta
	\end{lstlisting}
	

\subsection{Predicati sui gruppi}
Abbiamo visto come, tramite la clausola \texttt{group by}, le righe possano venire raggruppate in sottoinsiemi. Un'applicazione può aver bisogno di considerare solo i sottoinsiemi che soddisfano certe condizioni.\\
Se le condizioni che i sottoinsiemi devono soddisfare sono verificabili al livello delle singole righe, allora basta porre gli opportuni predicati come argomento della clausola \texttt{where}.\\
Se invece le condizioni sono delle condizioni di tipo aggregato, sarà necessario utilizzare un nuovo costrutto, la clausola \texttt{having}.\\\\
La clausola \texttt{having} descrive le condizioni che si devono applicare al termine dell'esecuzione di un'interrogazione che fa uso della clausola \texttt{group by}. Ogni sottoinsieme di righe costruito dalla \texttt{group by} fa parte del risultato dell'interrogazione solo se il predicato argomento della \texttt{having} risulta soddisfatto.\\\\
Il funzionamento dell'interrogazione è piuttosto semplice: si procede seguendo gli stessi passi descritti per le interrogazioni con \texttt{group by}. Dopo aver raggruppato le righe in base agli attributi specificati, viene valutato il predicato argomento della clausola \texttt{having}, selezionando le righe che lo soddisfano.\\\\
La sintassi permette anche la definizione di interrogazioni che presentano la clausola \texttt{having} senza una corrispondente clausola \texttt{group by}.\\
In questo caso l'intero insieme di righe è trattato come un unico raggruppamento, ma questo ha in generale un limitato campo di applicabilità, perché, se la condizione non è soddisfatta il risultato sarà vuoto.\\\\
Come la clausola \texttt{where}, anche la clausola \texttt{having} ammette come argomento un'espressione booleana su predicati semplici. I predicati semplici sono normalmente confronti tra il risultato della valutazione di un operatore aggregato e una generica espressione; sintatticamente è ammessa anche la presenza diretta degli attributi argomento della \texttt{group by}, ma è preferibile raccogliere tutte le condizioni su questi attributi nell'ambito della clausola \texttt{where}.\\
Per sapere quali predicati di un'interrogazione che fa uso del raggruppamento vanno dati come argomento della clausola \texttt{where} e quali come argomento della clausola \texttt{having}, basta rispettare il seguente criterio: solo i predicati in cui compaiono operatori aggregati devono essere argomento della clausola \texttt{having}.

\subsubsection{Esempio 7}
Consideriamo la seguente base dati:
	\begin{equation}\begin{aligned}
		IMPIEGATO (\underline{Nome, Cognome}, Dipartimento, Ufficio, Stipendio, Citta)\\
		DIPARTIMENTO (\underline{Nome}, Indirizzo, Citta)
	\end{aligned}\end{equation}
Estrarre i dipartimenti che spendono più di 100 in stipendi
	\begin{lstlisting}{Esempio 7}
		select
			I.Dipartimento,
			sum(I.Stipendio) as SommaStipendi
		from
			Impiegato as I
		group by
			I.Dipartimento
		having
			sum(Stipendio) > 100
	\end{lstlisting}
	
\subsubsection{Esempio 8}
Consideriamo la seguente base dati:
	\begin{equation}\begin{aligned}
		IMPIEGATO (\underline{Nome, Cognome}, Dipartimento, Ufficio, Stipendio, Citta)\\
		DIPARTIMENTO (\underline{Nome}, Indirizzo, Citta)
	\end{aligned}\end{equation}
Estrarre i dipartimenti per cui la media degli stipendi degli impiegati che lavorano nell'ufficio 20 è superiore a 25.
	\begin{lstlisting}{Esempio 8}
		select	
			I. Dipartimento
		from
			Impiegato as I
		where
			I.Ufficio = 20
		group by 
			I.Dipartimento
		having
			avg(I.Stipendio) > 25
	\end{lstlisting}

	

